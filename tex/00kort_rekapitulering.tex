\section{Short recap of second quantization for fermions and bosons}

Notation: $ \mu = $ set of quantum numbers that define a one-particle state.


\subsection{Many particle basis}
\newtheorem{theorem}{Ex}
\begin{theorem}


\begin{align*}
\mu &= (\vec{k}, \sigma):\text{Wave number, spin} \\
\mu &= (i, \sigma) : \text{Lattice point, spin} \\
\mu &= (n, i) : \text{Orbital, lattice point} \\
\end{align*}

\end{theorem}

A many-particle basis can be written $\ket{\phi} = \ket{n_{\mu}, n_{\nu}, \dots, n_{\mu_N}}$. Many particle states are built by combining many one-particle states, but where the one-particle states are not necessarily independent. If  \underline{one} of the set of quantum numbers, $\mu_i$, are changed, this \underline{scattering} will generally have consequences for the distribution of quantum numbers for the remaining sets $\{\mu_j\}_{j\ne i}$.
We generally imagine that many-particle states can be built as a linear combination of
$\ket{\phi}$'s;
\begin{equation}
\ket{\Psi} = \sum_{n_{\mu_1},\dots, n_{\mu_N}}\phi_{{\mu_1},\dots, n_{\mu_N}}\ket{{\mu_1},\dots, n_{\mu_N}}.
\end{equation}
A definite one-state vector $\ket{n_{\mu},\dots, n_{\mu_N}}$ can be demanded from a vacuum state (where there is no filled one-particle states) $\ket{0}$ via creation operators.
\begin{align*}
&\textbf{bosons}: &a_\mu^\dagger \\ 
&\textbf{fermions}: &c_\mu^\dagger
\end{align*}


A quanta in a one-particle state can be destroyed by the annihilation operators.

\begin{align*}
&\textbf{bosons}: &a_\mu \\ 
&\textbf{fermions}: &c_\mu
\end{align*}


These operators satisfy some commutation relations:
\begin{align}
&[a_\mu, a_{\nu}] & = [a_\mu^\dagger, a_{\nu}^\dagger] = 0 \\
&[a_\mu, a_{\nu}^\dagger] & = \delta_{\mu\,\nu} \\
&[A,B] &= AB-BA \\
&\{c_\mu^\dagger, c_{\nu}^\dagger\} & = \{c_\mu, c_{\nu}\} = 0 \\
&\{c_\mu, c_{\nu}^\dagger\} & = \delta_{\mu\,\nu} \\
&\{A,B\} &= AB + BA
\end{align}
These will automatically satisfy the Pauli principle as well, which gives \emph{symmetri/ \emph{antisymmetric}} solutions by exchange, dependent if the particles are bosons/fermions. 


\subsection{From classical formulation to second quantization of one-particle operators}

For one-particle operators we usually have a kinetic energy function on a form like
\begin{equation}
T = \sum_i T\left(\vec{r}_i, \vec{p}_i\right) = \sum_i T\left(\vec{r}_i, \diffp{}{r}\right)
\end{equation}
\begin{theorem}
External electrostatic potential:
\begin{equation}
T = \sum_i V_{\text{ext}}\left(\vec{r}_i\right)
\end{equation}
\end{theorem}
\begin{theorem}

Kinetic energy:

\begin{equation}
T = \sum_i \frac{p^2}{2m} = -\sum_i \frac{\hbar^2}{2m}\nabla_i^2
\end{equation}
\end{theorem}

\begin{theorem}
Crystal-potential:
\begin{equation}
T = \sum_i \sum_j v_{\text{cryst}} \left( \vec{r}_i, \vec{R}_j \right)
\end{equation}
\end{theorem}

Second quantization by an operator on this form can be written 
\begin{equation}
T = \sum_{\mu, \nu} T_{\mu\nu}c_\mu^\dagger c_{\nu},
\end{equation}
where
\begin{equation}
T_{\mu\,\nu} = \mel{\mu}{T\left(\vec{r}, \vec{p}\right)}{\nu}.	
\end{equation}

\textbf{Note: The matrix element of one-particle operators are determined by matrix elements in the Hilbert space of one-particle states.}

\subsection{From classical formulation to second quantization of two-particle operators}

Typically, we consider pair-potentials 
\begin{equation}
V = \sum_{i, j} V\left(\vec{r}_i, \vec{r}_j\right).
\end{equation}

\begin{theorem}
Exchange interaction of two charges
\begin{equation}
V = \frac{e^2}{2}\sum_{i\ne j} \frac{1}{|\vec{r}_i - \vec{r}_j|}
\end{equation}
\end{theorem}

The second quantization versions of these are

\begin{equation}
V = \sum_{\mu, \dots, \beta} V_{\mu\nu\alpha\beta}c_{\mu}^\dagger c_{\nu}^\dagger c_{\alpha}c_{\beta},
\end{equation}
where again
\begin{equation}
V_{\mu\nu\alpha\beta} = \mel{\mu\nu}{V\left(\vec{r}_i, \vec{r}_j\right)}{\beta\alpha}
\end{equation}

\textbf{Note: The matrix element of two-particle operators are determined by matrix elements in the Hilbert room of two-particle states.}

The Hamiltonian:
\begin{align}
H &= T + V \label{eq:hamiltonian} \\
T &= -\sum_i \frac{\hbar^2}{2m}\nabla_i^2
\end{align}
So far, we have just presented second quantization for fermion operators, but an equivalent statement will of course hold for the second quantization version of the Hamiltonian for an interacting, material, bosonic system, which has the same identical form as \eqref{eq:hamiltonian}. Notice that each term in $H$ has just as many $c_\mu^\dagger$ as $c_{\nu}$.


\begin{figure}
\centering
\begin{tikzpicture}

	\draw[thick, ->- = 0.5] (0,0) -- (1,1);
	\draw[thick, ->- = 0.5] (1, 1) -- (0,2);
	\draw[thick, dashed] (1, 1) -- (2.5, 1);
	\node[thick] at (2.5,1){$\cross$};
	\node[anchor = north west, thick] at (0.5, 0.5) {$\lambda_2$};
	\node[anchor = south west, thick] at (0.5, 1.5) {$\lambda_1$};
	\node[anchor = north west, thick] at (2, 1) {$v_{\lambda_1\lambda_2}$};
	\node[anchor = east] at (0, 1) {$T$:};
\end{tikzpicture}
\caption{Scattering from an external potential $v_{\mu\nu}c_{\mu}^\dagger c_{\nu}$}
\end{figure}


\begin{figure}
\centering
\begin{tikzpicture}

	\draw[thick, ->- = 0.5] (0,0) -- (1,1);
	\draw[thick, ->- = 0.5] (1, 1) -- (0,2);
	\draw[thick, dashed] (1, 1) -- (2.5, 1);
	\draw[thick, ->- = 0.5] (3.5, 0) -- (2.5, 1);
	\draw[thick, ->- = 0.5] (2.5, 1) -- (3.5, 2);
	
	\node[anchor = north] at (0, 0){$\lambda_3$};
	
	\node[anchor = north] at (0, 2){$\lambda_1$};	
	
	\node[anchor = north] at (3.5, 0){$\lambda_4$};

	\node[anchor = north] at (3.5, 2){$\lambda_2$};
	
	\node[anchor = east] at (0, 1) {$V$:};
\end{tikzpicture}
\caption{Exchange interaction between two particles.}
\end{figure}


\subsection{Statistical mechanics}

Assume that we know the spectrum $E_N^n$ for an interacting many-particle system, defined by a state $\ket{\psi_N}_n$, where $N$ is the number of particles in the system and $n$ is an index that indicates what excited state $\ket{\psi_N}_n$ the system is in. 
$\ket{\psi_N}$ is also assumed to be known, such that the matrix product of observables can be calculated: 
\begin{equation}
H\ket{\psi}_n = E_N\ket{\psi}_n.
\end{equation}

To do statistical mechanics, we need to introduce temperature. We do this by using the canonical partition function
\begin{equation}
\label{eq:partition_function}
Z_N = \sum_n\e^{-\beta E_N^n}.
\end{equation}

Note, in \eqref{eq:partition_function} we sum over states, \underline{not} the energy levels $E_N^n$.

\begin{align}
Z &= \sum_n {}_n\mel{\psi_N}{\e^{-\beta H}}{\psi_N}_n \nonumber \\
 &= \Tr\left(\e^{-\beta H}\right) = \Tr\left(S^{-1}S\e^{-\beta H}\right) \nonumber\\
 &= \Tr\left(S\e^{-\beta H}S^{-1}\right) \nonumber \\
 &= \sum_{n'}{}_{n'}\mel{\phi_N}{\e^{-\beta H}}{\phi_N}_{n'}. \label{eq:basis_invariant}
\end{align}

We see in \eqref{eq:basis_invariant} that we can use an \underline{arbitrary} basis to calculate the partition function. The most convenient basis is often a basis where the Hamiltonian is diagonal, but not always.

We write the statistical mean value of an operator as
\begin{align}
\expval{\hat{O}} &\equiv \frac{1}{Z}\Tr\left(\hat{O}\e^{-\beta H}\right) \nonumber\\
&= \frac{1}{Z} \sum_n {}_n\mel{\psi_N}{\hat{O}\e^{-\beta H}}{\psi_N}_n \nonumber \\
&= \frac{1}{Z} \sum_{n, n'} {}_n\expval{\hat{O}}{\psi_N}_{n'}\underbrace{{}_{n'}\expval{\e^{-\beta H}}{\psi_N}_n}_{\delta_{nn'}\e^{-\beta E_{n'}}}.
\end{align}
Thus, we have 
\begin{equation}
\label{eq:expval}
\expval{\hat{O}} = \frac{1}{Z}\sum_n \underbrace{{}_n\expval{\hat{O}}{\psi_N}_n}_{\text{QM matrix element}}\e^{-\beta E_N^n}.
\end{equation}
Notice how the temperature, $T$ only appears in the last factor in \eqref{eq:expval}.
Let us now consider the ground state ($n = 0$) in the low temperature limit with energy $E_0$ corresponding to the state $\ket{\psi_N}_0$.
\begin{align*}
\expval{\hat{O}} &\simeq \frac{1}{Z_{\beta = \infty}}\e^{-\beta E_0} {}_0\expval{\hat{O}}{\psi_N}_0 \\
&= \frac{\e^{-\beta E_0}}{\e^{-\beta E_0}}{}_0\expval{\hat{O}}{\psi_N}_0, 
\end{align*}

such that 
\begin{equation}
\expval{\hat{O}} \stackrel{\beta \rightarrow \infty}{=} {}_0\expval{\hat{O}}{\psi_N}_0.
\end{equation}

We now have a way to calculate the statistical mean value in the ground state at zero temperature. Let us now now assume that the energy spectrum is such that the ground state is separated from excited states by a  \underline{gap} (band insulators, semiconductors, superconductors). This way, we can express the excited state energies as 

\begin{equation}
E_N^1 = E_N^0 + \Delta_N
\end{equation}

such that 
\begin{equation}
E_N^2, E_N^3, \dots \ge E_N^1. 
\end{equation}

This way, we get from \eqref{eq:expval}

\begin{align}
\expval{\hat{O}} &= \frac{1}{Z}\sum_n {}_n\expval{\hat{O}}{\psi_N}_n\e^{-\beta E_N^n} \nonumber\\
&= \frac{\sum_n {}_n\expval{\hat{O}}{\psi_N}_n\e^{-\beta E_N^n}}{\sum_n\e^{-\beta E_N^n}} \nonumber\\
&= \cdots \nonumber\\
% Something fishy right here
&= \frac{{}_0\expval{\hat{O}}{\psi_N}_0\e^{-\beta E_N^0\left(1+\e^{-\beta\Delta}\dots\right)}}{\e^{-\beta E_N^0}\left(1+\e^{-\beta\Delta}\dots\right)}
\end{align}
and we find that as $\beta\Delta >> 1$, $\hat{O} \simeq {}_0\expval{\hat{O}}{\psi_N}_0$.
In semiconductors we find $\Delta \sim 10\mathrm{mev} \sim 1000\mathrm{K}$



















