\section[Coherent states]{Coherent states and introduction to Grassmann variables   }
Pages 10-17 in lecture notes.

\subsection{Coherent states}
A coherent state (both for fermions and bosons) is defined as an eigenstate to an annihilation operator

\begin{align}
	a_\mu \ket{\phi} &= \varphi_\mu \ket{\phi} & \textbf{Bosons} \label{eq:def_boson_coherent_state}\\
	c_\mu \ket{\psi} &= \xi_\mu\ket{\psi} & \textbf{Fermions} \label{eq:def_fermion_coherent_state}
\end{align}

Both $\ket{\psi}$ and $\ket{\phi}$ must contain a component with the least ($\ge 0$) quantum number (quant), but it is clear that neither $\ket{\psi}$ nor $\ket{\phi}$ can be states with a sharply defined number of particles. They are therefore also ``hard to destroy''. This also explains why we chose to define them as eigenstates of the annihilation operators, not the creation operators. We will get back to the creation of these coherent states. 

We will first look at the bosonic case:
\subsubsection*{Bosonic case}

\begin{equation}
a_\mu\ket{\phi} = \varphi_\mu\ket{\phi}
\end{equation}

\begin{align}
[a_\mu, a_{\nu}] = 0& &\nonumber \\
 &\Rightarrow \left(a_\mu a_{\nu} -a_{\nu}a_\mu\right)\ket{\phi} = 0 & \nonumber \\
&= \left(\varphi_\mu\varphi_{\nu} - \varphi_{\nu}\varphi_\mu\right)\ket{\phi} &\nonumber\\
&&\Rightarrow [\varphi_\mu, \varphi_\nu] = 0. \label{eq:coherent_commutator} 
\end{align}

Equation \eqref{eq:coherent_commutator} will always be satisfied if $\varphi_\mu \in \mathbb{C}$. \textbf{The eigenvalues to coherent boson states can be chosen as complex numbers. This is something we can state without knowing anything about how these states are constructed. }

\subsubsection*{Fermionic case}

\begin{equation}
c_\mu \ket{\psi} = \xi_\mu\ket{\psi}
\end{equation}

\begin{align}
\{c_\mu, c_\nu\} = 0 && \nonumber \\
&\Rightarrow\left(c_\mu c_\nu + c_\nu c_\mu\right)\ket{\psi} = 0 &\nonumber \\
&=\left(\xi_\mu\xi_\nu + \xi_\nu\xi_\mu\right)\ket{\psi} &\nonumber \\
&&\Rightarrow \{\xi_\mu,\xi_\nu\} = 0.\label{eq:coherent_anticommutator}
\end{align}

If $\xi_\mu\in\mathbb{C}$, \eqref{eq:coherent_anticommutator} will only be satisfied if $\{\xi_\mu\} = 0$, trivial eigenvalues. \textbf{The eigenvalues for coherent fermion states must be chosen as anti-commuting numbers, Grassmann-variables.}

\subsection{Grassmann variables}
\subsubsection*{Fundamentals}
	
Equation \eqref{eq:coherent_anticommutator} states the fundamental property of Grassmann variables, and it immediately follows that 
\begin{equation}
\xi_\mu^2 = 0,
\end{equation}
the squares of the Grassmann variables vanish!
Similarly we have that $\xi^n = \xi^2\xi^{n-2} = 0, n \ge 2$.
An arbitrary series expansion in Grassmann variables
\begin{align}
f(\xi) &= \sum_nc_n\xi^n \nonumber \\
&=c_0 + c_1\xi + \dots \nonumber\\
&= c_0 + c_1\xi
\end{align}
is linear. We can also consider $f(\xi^*) = c_0 + c_1\xi^*$, where $\left(\xi^*\right)^* = \xi$.
An arbitrary function of $\xi, \xi^*$ can be written on the forms
\begin{align}
\label{eq:grassmann_function}
A\left(\xi, \xi^*\right) &= c_0 + c_1\xi + c_2\xi^*+c_3\xi\xi^* \\
&= c_0 + c_1\xi + c_2\xi^*+d_3\xi^*\xi
\end{align}

We will now look into some of the properties of functions of Grassmann variables.
\subsubsection*{Differentiation}

Differentiation with respect to Grassman variables follows
\begin{align}
&\frac{\partial \xi}{\partial \xi} = 1 & \frac{\partial \xi}{\partial \xi^*} = 0   \\[2ex]
&\frac{\partial \xi^*}{\partial\xi} = 0 & \frac{\partial \xi^*}{\partial \xi^*} = 1  \\[2ex]
&\pdv{\left(\xi\xi^*\right)}{\xi} = \xi^* &  \\[2ex]
&\pdv{\left(\xi\xi^*\right)}{\xi^*} = -\pdv{\left(\xi^*\xi\right)}{\xi^*} = -\xi  \\[2ex]
&\pdv{f\left(\xi\right)}{\xi^*} = 0 &\\[2ex]
&\pdv{f\left(\xi\right)}{\xi} = c_1 = \pdv{f\left(\xi^*\right)}{\xi^*}, &
\end{align}
and for functions defined as in \eqref{eq:grassmann_function}, we have

\begin{align}
\label{eq:grassmann_derivation}
\pdv{\xi}A\left(\xi, \xi^*\right) &= c_1 + c_3\xi^* = c_1 - d_3\xi^* \\
\pdv{\xi^*}A\left(\xi, \xi^*\right) &= c_2 - c_3\xi = c_2 + d_3\xi.
\end{align}

\subsubsection*{Integration}
Integrating with respect to Grassmann variables are motivated from the properties of ``normal'' Riemann integrals, that if $f\left(x = \pm \infty\right) = 0$, then
\begin{equation}
\int_{-\infty}^{\infty}\dx \dv{f}{x} = 0.
\end{equation}

Equivalently we define
\begin{align}
&\int\dxi\cdot 1 =\int\dxi\dv{\xi}{\xi} = 0 \\
&\int\dxi^* = 0,
\end{align}

in other words, the integral of a total differential is zero.

\begin{equation}
\int\dxi\xi = \int\dxis\xi^* = 1
\end{equation}
is a ``normalization'' criteria. These relations define what we mean by Grassmann-integration.

Now, we have 
\begin{align}
\int\dxi f\left(\xi\right) &= c_1 \\
\pdv{f}{\xi} &= c_1 \\
\int\dxi A\left(\xi, \xi^*\right) &= \int\dxi \left(x_0+c_1\xi+c_2\xi^*+c_3\xi\xi^*\right)\nonumber \\
&=c_1 + c_3\xi^*. \label{eq:grassmann_integral}
\end{align}
As we can see by comparing \eqref{eq:grassmann_integral} with \eqref{eq:grassmann_derivation}, ``integration'' $=$ ``derivation''. The somewhat hand wavy definition of integration is motivated by the fact that it gives results that reminds us about results from the theory for complex functions. 

\subsubsection*{The number operator}
Generally, we have that 

\begin{align}
&a_\mu^\dagger\ket{\phi} \ne \ket{\phi} & c_\mu^\dagger\ket{\xi} \ne \ket{\xi}, 
\end{align}

and so

\begin{align}
a_\mu^\dagger a_\mu\ket{\phi} &\ne N_\mu^\phi\ket{\phi} \\ 
c_\mu^\dagger c_\mu\ket{\xi} &\ne N_\mu^\xi\ket{\xi}.
\end{align}
\textbf{The coherent states are \underline{not} eigenstates of the counting operator. $\ket{\phi}, \ket{\xi}$ are not states with a fixed number of ``quants''.}

\subsubsection*{Algebra}

Consider a vector space with the following additional properties:

\begin{align*}
&1) & (xy)z =x(yz) \\
&2) & x(y+z) = xy+xz \\
&3)  & (x+y)z =xz+yz \\
&4) & \alpha xy = x\alpha y.
\end{align*}

In Abelian algebra, $xy = yx$, while in Grassmann algebra $xy = -yx$. \underline{Complex numbers} are generators for the Abelian algebra over the field $\mathbb{G}$ of commuting numbers. \underline{Grassmann numbers} are generators for the algebra over the field $\mathbb{G}$ of anticommuting numbers. 

