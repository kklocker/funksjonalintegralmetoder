\section{Interacting one-dimensional fermionic systems}
%(page 233 pdf, 182 notes)
%\subsection{Intro} % Snorre

\subsection[Model]{1D interacting model}

We model the system by $\Ha = H_0 + H_I$ as usual. Consider first $H_0$
\begin{align*}
	\Ha_0 &= \sum_{i,j,\sigma}t_{ij}\cd_{i\sigma}c_{j\sigma} \\
	&\implies \sum_{k \sigma}\varepsilon_k\cd_{k\sigma}c_{k\sigma}
\end{align*}

\begin{figure}
	\centering
	\begin{tikzpicture}
	\begin{axis}[
	legend style={at={(1,0.4)}},
	ytick = {1},
	ytick style={draw=none},
	yticklabels = {},
	xtick = {-3.14159,3.14159},
	xticklabels ={$-\pi$,$\pi$},
	xlabel = \large $k$,
	ylabel = \large $\varepsilon_k$,
	axis lines = center,
	ymin = -1.5,
	ymax = 1.5,
	xmax = 4,
	xmin = -4,
	%x label style={at={(axis description cs:1,0.1)}, anchor = west},
	%y label style={at={(axis description cs:0.15,1)},rotate=-90,anchor=south},
	%xlabel = $x$,
	%sylabel = {$f(x)$},
	]
	
		
		\addplot[name path = A,domain = -1.57:1.57, samples = 200, very thin, black]{(-cos(deg(x)))};
		\addplot[domain = -3.14:3.14, samples = 200, thick, red]{(-cos(deg(x)))};
		\addplot[name path = B, domain = -1.57:1.57] {0*x};
		
		\addplot[blue!10!white] fill between[of = A and B];
		\draw[thick, black, dashed] (axis cs:-1.57,0) -- node[anchor= north east]{\large $-k_F$}  (axis cs:-1.570,-0.5);
		\draw[thick, black, dashed] (axis cs:1.57,0) -- node[anchor= north west]{\large $k_F$}  (axis cs:1.570,-0.5);
	\end{axis}
\end{tikzpicture}
	\caption{Dispersion relation. The fermi surface consists of two points}
	\label{fig:disp_rel}
\end{figure}

We are mainly interested in the low energy physics for this problem; We want to describe $\varepsilon_k$ in the vicinity of the Fermi level. In this case, it is especially simple as it only consists of two points, as seen in Figure \ref{fig:disp_rel}.

Right-moving: $k>0$ and left-moving: $k<0$. For both of which we have  $\varepsilon_k\simeq V_F(|k| - k_F)$ and $|k-k_F|<<k_F$. $\varepsilon_k$ is here measured relative the Fermi level.
For every electron spin $\sigma$ we have one type of righ-moving and one type of left-moving fermions, seen in Figure \ref{fig:lr_fermions}.

\begin{figure}
	\centering
	\input{tex/img/rl_fermions}
	\caption{The right- and left-moving fermions of interest.}
	\label{fig:lr_fermions}
\end{figure}

\begin{align}
\label{eq:h0_rl}
\begin{split}
\Ha_0 &\simeq \sum_{k,\sigma}V_F(k-k_F)\cd_{1k\sigma}c_{1k\sigma} \\
&+\sum_{k,\sigma}V_F(-k-k_F)\cd_{2k\sigma}c_{2k\sigma},
\end{split}
\end{align}
where $\cd_{1k\sigma}$ creates a right-moving fermion and $\cd_{2k\sigma}$ creates a left-moving fermion etc.
NB: \ref{eq:h0_rl} is defined only at $|\pm k-k_F|<<k_F$!
We write this as
\begin{align}
\begin{split}
\Ha=\sum_{k,\sigma}&\left[\e^{-\alpha|k-k_F|}V_F(k-k_F)\cd_{1k\sigma}c_{1k\sigma}\right. \\
&+\left.\e^{-\alpha|k-k_F|}V_F(-k-k_F)\cd_{2k\sigma}c_{2k\sigma}\right].
\end{split}
\end{align}
Here, $\alpha>0$ is a cutoff-parameter to limit the $k$-summation to those who are close to Fermi level, but in such a way that we can sum over all $k$.
Introduce the real space fermion operators
\begin{align}
\psi_{1\sigma}(x) & =\int\frac{\dd k}{2\pi}c_{1k\sigma}\e^{ikx}
\end{align}






















\subsection[Free propagators]{Free fermion propagators} % pdf: 245 / notes: 192
\subsection[Interacting propagator]{Propagators in  the interacting case} % pdf:248 / notes 196
\subsection{The phase factor} % 258 / 203
\subsection{The bosonic excitation spectrum} % 270 / 212a
\subsection{Impulse distribution} % 277 / 212f  Kanskje ta side 212a - 212g i samme subsection?
\subsection{``Summary''} % 279 / 213
