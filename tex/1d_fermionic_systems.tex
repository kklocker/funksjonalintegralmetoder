\section{Interacting one-dimensional fermionic systems}
%(page 233 pdf, 182 notes)
%\subsection{Intro} % Snorre

\subsection[Model]{1D interacting model}

We model the system by $\Ha = H_0 + H_I$ as usual. Consider first $H_0$
\begin{align*}
	\Ha_0 &= \sum_{i,j,\sigma}t_{ij}\cd_{i\sigma}c_{j\sigma} \\
	&\implies \sum_{k \sigma}\varepsilon_k\cd_{k\sigma}c_{k\sigma}
\end{align*}

\begin{figure}
	\centering
	\begin{tikzpicture}
	\begin{axis}[
	legend style={at={(1,0.4)}},
	ytick = {1},
	ytick style={draw=none},
	yticklabels = {},
	xtick = {-3.14159,3.14159},
	xticklabels ={$-\pi$,$\pi$},
	xlabel = \large $k$,
	ylabel = \large $\varepsilon_k$,
	axis lines = center,
	ymin = -1.5,
	ymax = 1.5,
	xmax = 4,
	xmin = -4,
	%x label style={at={(axis description cs:1,0.1)}, anchor = west},
	%y label style={at={(axis description cs:0.15,1)},rotate=-90,anchor=south},
	%xlabel = $x$,
	%sylabel = {$f(x)$},
	]
	
		
		\addplot[name path = A,domain = -1.57:1.57, samples = 200, very thin, black]{(-cos(deg(x)))};
		\addplot[domain = -3.14:3.14, samples = 200, thick, red]{(-cos(deg(x)))};
		\addplot[name path = B, domain = -1.57:1.57] {0*x};
		
		\addplot[blue!10!white] fill between[of = A and B];
		\draw[very thick, black, dashed] (axis cs:-1.57,0) -- node[anchor= north east]{\large $-k_F$}  (axis cs:-1.570,-0.7);
		\draw[very thick, black, dashed] (axis cs:1.57,0) -- node[anchor= north west]{\large $k_F$}  (axis cs:1.570,-0.7);
	\end{axis}
\end{tikzpicture}
	\caption{Dispersion relation. The fermi surface consists of two points}
	\label{fig:disp_rel}
\end{figure}

We are mainly interested in the low energy physics for this problem; We want to describe $\varepsilon_k$ in the vicinity of the Fermi level. In this case, it is especially simple as it only consists of two points, as seen in Figure \ref{fig:disp_rel}.

Right-moving: $k>0$ and left-moving: $k<0$. For both of which we have  $\varepsilon_k\simeq v_F(|k| - k_F)$ and $|k-k_F|<<k_F$. $\varepsilon_k$ is here measured relative the Fermi level.
For every electron spin $\sigma$ we have one type of righ-moving and one type of left-moving fermions, seen in Figure \ref{fig:lr_fermions}.

\begin{figure}
	\centering
	\begin{tikzpicture}
	\begin{axis}[
	legend style={at={(1,0.4)}},
	ytick = {1},
	ytick style={draw=none},
	yticklabels = {},
	xtick = {-3.14159,3.14159},
	xticklabels ={$-\pi$,$\pi$},
	xlabel = \large $k$,
	ylabel = \large $\varepsilon_k$,
	axis lines = center,
	ymin = -1.5,
	ymax = 1.5,
	xmax = 4,
	xmin = -4,
	%x label style={at={(axis description cs:1,0.1)}, anchor = west},
	%y label style={at={(axis description cs:0.15,1)},rotate=-90,anchor=south},
	%xlabel = $x$,
	%sylabel = {$f(x)$},
	]
	
		
		\addplot[name path = A,domain = -3.14:3.14, samples = 200, dashed, black]{(-cos(deg(x)))};
		
		\addplot[domain = -1.77:-1.37, samples = 200,very thick, red]{(-cos(deg(x)))};
		\addplot[domain = 1.37:1.77, samples = 200,very thick, red]{(-cos(deg(x)))};
		
		
		\draw[thick, black, dashed] (axis cs:-1.57,0) -- node[anchor= north east]{\large $-k_F$}  (axis cs:-1.570,-0.5);
		\draw[thick, black, dashed] (axis cs:1.57,0) -- node[anchor= north west]{\large $k_F$}  (axis cs:1.570,-0.5);
	\end{axis}
\end{tikzpicture}
	\caption{The right- and left-moving fermions of interest.}
	\label{fig:lr_fermions}
\end{figure}

\begin{align}
\label{eq:h0_rl}
\begin{split}
\Ha_0 &\simeq \sum_{k,\sigma}v_F(k-k_F)\cd_{1k\sigma}c_{1k\sigma} \\
&+\sum_{k,\sigma}v_F(-k-k_F)\cd_{2k\sigma}c_{2k\sigma},
\end{split}
\end{align}
where $\cd_{1k\sigma}$ creates a right-moving fermion and $\cd_{2k\sigma}$ creates a left-moving fermion etc.
NB: \ref{eq:h0_rl} is defined only at $|\pm k-k_F|<<k_F$!
We write this as
\begin{align}
\begin{split}
\Ha=\sum_{k,\sigma}&\left[\e^{-\alpha|k-k_F|}v_F(k-k_F)\cd_{1k\sigma}c_{1k\sigma}\right. \\
&+\left.\e^{-\alpha|k-k_F|}v_F(-k-k_F)\cd_{2k\sigma}c_{2k\sigma}\right].
\end{split}
\end{align}
Here, $\alpha>0$ is a cutoff-parameter to limit the $k$-summation to those who are close to Fermi level, but in such a way that we can sum over all $k$.
Introduce the real space fermion operators
\begin{align}
\psi_{1\sigma}(x) & =\int\frac{\dd k}{2\pi}c_{1k\sigma}\e^{ikx} \\
\psi_{2\sigma}(x) & =\int\frac{\dd k}{2\pi}c_{1k\sigma}\e^{-ikx},
\end{align}
where the sums over $k$ is calculated close to Fermi-level.
Consider now the free fermion propagators in real space at equal times $(\tau = \tau')$

\begin{align*}
\mathcal{G}_1(x) &= -i\ev{:\psi_{1\sigma}(x)\psi_{1\sigma}^\dagger(x'):}\\
&= -i\int\frac{\dd k}{2\pi}\int\frac{\dd k'}{2\pi}\e^{ik(x-x')}\underbrace{\ev{:c_{1k\sigma}\cd_{1k\sigma}:}}_{=-\delta_{k,k'}n_k} \\
&=i\int_{k<k_F}\frac{\dd k}{2\pi}\e^{-\alpha|k-k_F|}\e^{ik(x-x')} \\
&= i\int_{q<0}\frac{\dd q}{2\pi}\e^{-\alpha|q|}\e^{i(q+k_F)(x-x')} \\
&=i\e^{ik_F(x-x')}\int_{-\infty}^{0}\frac{\dd k}{2\pi}\e^{-\alpha|k| + ik(x-x')}\\
&= i\frac{\e^{ik_F(x-x')}}{2\pi}\int_0^\infty\dd k\,\e^{-k(\alpha + i(x-x'))} \\
&=i\frac{\e^{ik_F(x-x')}}{2\pi}\frac{1}{\alpha + i(x-x')},
\end{align*}
or, to summarize, as $\alpha \rightarrow 0$ (ignoring the short-range cutoff)
\begin{align}
\mathcal{G}_1 &= \frac{1}{2\pi}\frac{\e^{ik_F(x-x')}}{x-x'} \\
\mathcal{G}_2 &= -\frac{1}{2\pi}\frac{\e^{-ik_F(x-x')}}{x-x'}.
\end{align}
Notice the simple poles in the Green's functions. Now, by the $\psi_{1\sigma}$ and $\psi_{2\sigma}$-operators, we have

\begin{equation}
\Ha_0 = v_F\int_{0}^{L}\dd x\left[\psi_{1\sigma}^\dagger\left(-i\partial_x -k_F\right)\psi_{1\sigma} + \psi_{2\sigma}^\dagger\left(i\partial_x - k_F\right)\psi_{2\sigma}\right].
\end{equation}
Defining $\psi_\sigma = \begin{psmallmatrix*} \psi_{1\sigma} & \psi_{2\sigma}
\end{psmallmatrix*}^T$ lets us rewrite this as
\begin{equation}
\label{eq:imp_def_a0}
\Ha_0 = v_F\sum_\sigma\int_{0}^{L}\dd x\,\psi_\sigma^\dagger
\underbrace{\begin{pmatrix}
-i\partial_x - k_F & 0 \\
0 &  i\partial_x - k_F 
\end{pmatrix}}_{A_0}\psi_\sigma
\end{equation}

We now add an interaction term to the theory, on the form
\begin{align*}
\Ha_I &= \sum_{\sigma,\sigma'}\int\dd x\int\dd y\underbrace{\psi_{1\sigma}^\dagger(x)\psi_{1\sigma}(x)}_{\rho_{1\sigma}(x)}v(x-y)\underbrace{\psi_{2\sigma}^\dagger(y)\psi_{2\sigma'}(y)}_{\rho_{2\sigma'}(y)} \\
&= \int \dx\dy \rho_1(x)v(x-y)\rho_2(y).
\end{align*}
We do not consider terms of the types $\rho_1\rho_1, \rho_2\rho_2$; these only contributes to the renormalization of $v_F$. $\rho_1\rho2$ will, however, give rise to more intersting effects.

Therefore, we want to study the theory of

\begin{align}
\begin{split}
\Ha &= v_F \sum_\sigma \int\dx\psi_\sigma^\dagger A_0\psi_\sigma \\
&+ \sum_{\sigma,\sigma'}\int \dx\dy \rho_{1\sigma}(x)v(x-y)\rho_{2\sigma'}(y).
\end{split}
\end{align}

For simplicity, we will in the continuation consider spinless fermions. A spin treatment will be given later. 

\begin{equation}
\Ha = v_F \int\dx\psi^\dagger A_0\psi+ \int \dx\dy \rho_{1}(x)v(x-y)\rho_{2}(y).
\end{equation}
A spinless system is the same (or equivalent) to a completely spin-polarized problem. We can write down the partition function by applying our normal rules;
\begin{align}
\Z &= \int\D\psi_1^\dagger\D\psi_1\D\psi_2^\dagger\D\psi_2\e^{\Sa} \\
\begin{split}
\Sa &= -\int\dx \int_0^\beta\dd\tau\left(\psi^\dagger\partial_\tau\psi + v_F\psi^\dagger A_0\psi\right) \\
&- \int\dx\int\dy \int_0^\beta\dd\tau\rho_{1}(x)v(x-y)\rho_{2}(y)
\end{split}
\end{align}
Let us simplify the model some more by only considering constant \textit{contact}-interaction between fermions 
\begin{equation}
	v(x-y) = g\delta(x-y); \quad g>0.
\end{equation}

\begin{align*}
	\Sa &= \Sa_0 + \Sa_I \\
	\Sa_0 &= -\int_0^L\dx\int_0^\beta\dd\tau\psi^\dagger\left(\partial_\tau+ v_FA_0\right)\psi \\
	\Z_0 &= \e^{\Tr\ln(\partial_\tau + v_FA_0)} \\
	&= \e^{\int_0^L\dx\int_0^\beta\dd\tau\tr\ln\left(\partial_\tau+ v_FA_0\right)} \\
	\Sa_I &= -g\int\dx\int\dd\tau\rho_1(x)\rho_2(x)
\end{align*}
Now we perform a Hubbard-Stratonovich decoupling og the interaction term in the following way;
\begin{align*}
\int\D\pdag\D\p\,\e^{-(\pdag + g\rho_2)(\p + g\rho_1)g^{-1}} &= \text{const} \\
&= \e^{-\Tr\ln(g^{-1})} = \e^{\ln g} = g.
\end{align*}
Correspondingly,we have
\begin{align*}
\int\D\pdag\D\p\,&\e^{-\int\dx\dd \tau\left(g^{-1}\p^*\p+i\p^*\rho_1 + i\p^*\rho_2\right)} \\
&= \int\D\pdag\D\p\,\e^{-\int\dx\dd \tau\left[g^{-1}(\p^* + ig\rho_2)(\p + ig\rho_1) + g\rho_1\rho_2\right]} \\
&= \e^{-\int\dx\dd \tau g\rho_1\rho_2}\cdot \text{const}.
\end{align*}
The constant is irrelevant and is safely ignored.
\begin{align}
\e^{\Sa_I} &= \int\D\p^*\D\p\e^{-\int\dx\int_0^\beta\left(g^{-1}\p^*\p+i\p^*\rho_1 + i\p\rho_2\right)} \\
\Z &= \int\D\p^*\D\p\D\psi_1^\dagger\D\psi_1\D\psi_2^\dagger\D\psi_2\e^{\Sa[\psi^\dagger,\psi,\p^*\p]}
\end{align}
\begin{equation}
\label{eq:action_1d}
\Sa= -\int_0^L\dx \int_0^\beta \dd\tau \left\{\frac{\p^*\p}{g} +  \psi^\dagger A\psi\right\},
\end{equation}
where\footnote{There is some ambiguity here. In the notes, $\vb{I}$ is denoted $\vec{T}$, so it should be checked that it is the identity operator. We should have a term with $\psi^\dagger\partial_\tau\psi$, so the identity makes sense.}
\begin{equation}
	A \equiv v_FA_0 + \vb{I}\partial_\tau + i
	\begin{pmatrix}
	\p^* & 0 \\
	0 & \p
	\end{pmatrix} = -\mathcal{G}^{-1},
\end{equation}
with $A_0$ defined in \eqref{eq:imp_def_a0}, and
\begin{equation}
	\mathcal{G}(\partial_\tau, x) = -\ev{\psi\psi^\dagger}.
\end{equation}

Note that there is no terms $\sim \p^*\partial_\tau\p$ in \eqref{eq:action_1d}, because the $\p$-field is absent in the Hamiltonian, but is used to decouple the interaction first after the partition function is written down. If we use the spinor-components reduces to 
\begin{equation}
\mathcal{G}(\partial_\tau,x) = -\begin{pmatrix}
\ev{\psi_1\psi_1^\dagger} & 0\\
0 & \ev{\psi_2\psi_2^\dagger}
\end{pmatrix}.
\end{equation}
Since the matrix is diagonal, we can immidiately write down 
\begin{align}\label{eq:greeenss}
\begin{split}
\mathcal{G}_1 &= -(-iv_F\partial_x + \partial_\tau - v_Fk_F + i\p^*)^{-1} \\
\mathcal{G}_2 &= -(iv_F\partial_x + \partial_\tau - v_Fk_F + i\p)^{-1}
\end{split}
\end{align}
The Greens functions \eqref{eq:greeenss} are free Greens functions in \underline{fixed} bosonic background field.
Rewriting these as differential equations, we get
\begin{align}\label{eq:diff_eq_greens}
\begin{split}
(-iv_F\partial_x + \partial_\tau - v_Fk_F + i\p^*)\mathcal{G}_1 &= -\delta(x)\delta(\tau) \\
 (iv_F\partial_x + \partial_\tau - v_Fk_F + i\p)\mathcal{G}_2 &= -\delta(x)\delta(\tau).
\end{split}
\end{align}
NB! These are limited to the one-dimensional case, and are complicated because of the dynamics of $\p(x,\tau), \p^*(x, \tau)$.

\subsection[Free propagators]{Free fermion propagators} % pdf: 245 / notes: 192

The free fermion propagators for our problem is given by

\begin{align}\label{eq:diff_eq_greens}
\begin{split}
(-iv_F\partial_x + \partial_\tau - v_Fk_F )\mathcal{G}_1^0 &= -\delta(x)\delta(\tau) \\
(iv_F\partial_x + \partial_\tau - v_Fk_F)\mathcal{G}_2^0 &= -\delta(x)\delta(\tau).
\end{split}
\end{align}
We want to compute $\mathcal{G}_1,\mathcal{G}_2$ because of all the information these provide to the low-energy physics for the one-dimensional interacting electron gas. Recall the usage as diagnostic tools in the Fermi-liquid case. Now we do a Fourier transform of the equations to obtain
\begin{align}
\nonumber
(iv_Fk+ i\omega_n- v_Fk_F )\mathcal{G}_1^0&= -1 \\
\implies \mathcal{G}_1^0(k,i\omega_n) = \frac{1}{i\omega_n - v_F(k-k_F)} &= \frac{1}{i\omega_n - \varepsilon_1(k)}\\
\nonumber
(-iv_Fk+ i\omega_n- v_Fk_F )\mathcal{G}_2^0&= -1 \\
\implies \mathcal{G}_2^0(k,i\omega_n) = \frac{1}{i\omega_n - v_F(-k-k_F)} &= \frac{1}{i\omega_n - \varepsilon_2(k)},
\end{align}

which are free-form propagators. How does these look like in real space?
Impose, again, a cutoff to make the integral convergent, and set the cutoff-parameter to zero at the end.
\begin{align*}
\mathcal{G}_1^0(x, \tau) &= \int_{-\infty}^\infty\frac{\dd k}{2\pi}\frac{1}{\beta}\sum_{\omega_n}\frac{\e^{ikx - i\omega_n\tau}}{i\omega_n-v_F(k-k_F)}\cdot\underbrace{\e^{-\alpha|k-k_F|}}_{\text{Cutoff-parameter}} \\
&= \int_{-\infty}^\infty\frac{\dd k}{2\pi}\e^{-\alpha|k-k_F| + ikx}f(v_F(k-k_F))\e^{-v_F(k-k_F)\tau}\\
&=\e^{ik-Fx}\int_{-\infty}^0\frac{\dd k}{2\pi}\e^{-\alpha|k| + ikx-v_Fk\tau} \\
&=\frac{\e^{ik_Fx}}{2\pi}\int_0^\infty\dd k\,\e^{-k(\alpha + ix-v_F\tau)} \\
&= \frac{\e^{ik_Fx}}{2\pi}\frac{1}{ix - v_F\tau + \alpha},
\end{align*}
$\alpha = 0^+$. Now let $\tau\rightarrow it$ such that $\mathcal{G}_1^0(x,\tau)\rightarrow i\mathcal{G}_1^0(x, t)$.
\begin{align}
\begin{split}
	\mathcal{G}_1^0(x, t) &=  i\frac{\e^{ik_Fx}}{2\pi}\frac{1}{ix-iv_Ft+\alpha} \\
	&=\frac{\e^{ik_Fx}}{2\pi}\frac{1}{x-v_Ft-i\alpha}
\end{split}
\end{align}
We see that $\mathcal{G}_1^0$ has a pole at $x = v_Ft$. This is classicall linear motion (right-moving). A very similar approach\footnote{The only difference is in the substitution $k' = -k-k_F$ before integrating.} can be used to calculate $\mathcal{G}_2^0$, and leads to 
\begin{equation}
	\mathcal{G}_2^0=\frac{\e^{-ik_Fx}}{2\pi}\frac{1}{-x-v_Ft-i\alpha}, 
\end{equation}
with a pole at $x = -v_Ft$, classical linear motion towards the left.
Therefore, Fermi-liquid propagators also have a simple pole-structure with a simple interpretation in real space. 


\subsection[Interacting propagator]{Propagators in  the interacting case} % pdf:248 / notes 196

The theory is
\begin{align}
\Z &= \int\D\psi^\dagger\D\psi\D\p^*\D\p\e^{\Sa} \\
\Sa &=-\int_0^L\dx\int_0^\tau\dd\tau \frac{\p^*\p}{g}-\int_0^L\dx\int_0^L\dy\int_0^\beta\dd\tau\,\psi^\dagger(x,\tau)A(x,y,\tau)\psi(y, \tau),
\end{align}
\begin{equation*}
A = \begin{pmatrix}
D_1 + i\p^* & 0\\
0 & D_2+i\p
\end{pmatrix}
\end{equation*}

\begin{align*}
D_1 = -&iv_F\partial_x+\partial_\tau-v_Fk_F \\
D_2 =\quad&iv_F\partial_x+\partial_\tau-v_Fk_F
\end{align*}
which is a free fermion system coupled to an external bosonic field $\p$ which fluctuates. Integrating out the fermionic field gives a purely bosonic theory of effective action
\begin{align}
\Z &= \int\D\p^*\D\p\e^{\Sa_{\text{eff}}[\p^*,\p]} \\
\label{eq:s_eff}
\Sa_{\text{eff}} &= \int_{0}^{\beta}\dd\tau\int_{0}^{L}\dx\frac{\p^*\p}{g} + \Tr\ln(D_1+i\p_1) + \Tr\ln(D_2 +i\p_2),
\end{align}
where $\p_1 = \p^*, \p_2 = \p$. 
\begin{align*}
-\mathcal{G}_1^{-1} &= D_1+i\p_1 \\
-\mathcal{G}_2^{-1} &= D_2 +i\p_2
\end{align*}
The propagators above are matrices with $x, \tau;x',\tau'$-indices. $D_i$ includes non local differential operators. We have Green's functions for free fermions coupled to an external bosonic field
\begin{align*}
\mathcal{G}_1 &= \mathcal{G}_1(x, \tau;x', \tau, \p_1) \\\mathcal{G}_2 &= \mathcal{G}_2(x, \tau;x', \tau, \p_2).
\end{align*}

\begin{align}
\label{eq:gre_1}
(D_1+i\p_1)\mathcal{G}_1 &= -\delta(x-x')\delta(\tau-\tau') \\
\label{eq:gre_2}
(D_2 +i\p_2)\mathcal{G}_2 &= -\delta(x-x')\delta(\tau-\tau')
\end{align}

$D_1, D_2$ are 1. order differential operators. Assume now that $x\ne x',\tau\ne\tau'$ such that the right hand side in the equations is zero.
\[D_i\mathcal{G}_i = -i\p_i\mathcal{G}_i \implies \frac{D_i\mathcal{G}_i}{\mathcal{G}_i} = -i\p_i,\quad i = 1,2\]
Now integrate
\begin{align}
\int_{x',\tau'}^{x,\tau}\frac{D_i\gre_i}{\gre_i} &= -i\int_{x',\tau'}^{x,\tau}\p_i \\
\ln\gre_i\bigg{|}_{x',\tau'}^{x,\tau} &= -i\left[f_i(x,\tau,\p_i) - f_i(x',\tau',\p_i)\right]
\end{align}



\[\gre_i = A\e^{-i(f_i(x,\tau)-f_i(x',\tau'))}\]
$\gre_i = \gre_i(x,\tau;x',\tau',\p_i)$. If we now let $\p_i\rightarrow 0 \implies f_i =0$, we must have that $A = \gre_i^0(x-x',\tau-\tau')$. This motivates the following ansatz for the Green's functions in the interacting case:
\begin{equation}
\gre_i(x,\tau;x',\tau',\p_i) = \gre_i^0(x-x',\tau-\tau')\e^{-i\left[f_i(x,\tau,\p_i) - f_i(x',\tau',\p_i)\right]}
\end{equation}

\[D_i\gre_i^0 = -\delta(x-x')\delta(\tau-\tau')\]
The ansatz for $\gre_i$ is now inserted into \cref{eq:gre_1,eq:gre_2}.
$D_i = d_i - V_Fk_F, d_i = \pm iv_F\partial_x + \partial_\tau$.
\begin{align*}
\gre_i(x,\tau;x'\tau') &=(d_i - v_Fk_F+i\p_i)\gre_i^0\e^{-i(f_i-f_i')}\\ &=\left[(d_i-v_Fk_F)\gre_i^0\right]\e^{-i(f_i-f_i')} + i\p_i\gre_i^0\e^{-i(f_i-f_i')} -id_if_i\gre_i^0\e^{-i(f_i-f_i')} \\
&= -\delta(x-x')\delta(\tau-\tau')
\end{align*}
Now use
\begin{align*}
\left[(d_i-v_Fk_F)\gre_i^0\right]\e^{-i(f_i-f_i')} &= -\delta(x-x')\delta(\tau-\tau')\e^{-i(f_i-f_i')}\\&=\delta(x-x')\delta(\tau-\tau');\quad x\ne x',\tau=\tau' \\
\implies  -id_if_i &= -i\p_i
\end{align*}
\begin{align}
-i(-iv_F\partial_x+\partial_\tau)f_1 &= -i\p_1 \\
-i(iv_F\partial_x+\partial_\tau)f_2 &= -i\p_2
\end{align}
To continue, we fourier transform these equations. Since $\p_i$ are bosonic fields, $f_i$ are aswell, and Fourier-transform with bosonic matsubara frequencies \[\omega_\nu = \frac{2\pi\nu}{\beta}\quad;\nu = 0,\pm1,\dots\].
\begin{align}
f_i(x,\tau) &= \frac{1}{\beta}\int\frac{\dd k}{2\pi}\sum_{\omega_\nu}f_i(k,\omega_\nu)\e^{i(kx-\omega_\nu\tau)}\\
\label{eq:fourier_transformed_1d_fields}
\p_i(x,\tau) &= \frac{1}{\beta}\int\frac{\dd k}{2\pi}\sum_{\omega_\nu}\p_i(k,\omega_\nu)\e^{i(kx-\omega_\nu\tau)}
\end{align}
Inserting these gives
\begin{align}
\nonumber
-i(v_Fk-i\omega_\nu)f_i(k,i\omega_\nu) &= -i\p_i(k,i\omega_\nu) \\
\implies f_1(k,i\omega_\nu) 
&= -\frac{\p_1(k,i\omega_\nu)}{i\omega_\nu - v_Fk}, \\
f_2(k,i\omega_\nu) &= -\frac{\p_2(k,i\omega_\nu)}{i\omega_\nu + v_Fk}
\end{align}

Transforming back to real space, using
\begin{equation}
\label{eq:fourier_transformed_field_2}
f_i(x,\tau) = -\frac{1}{\beta}\int\frac{\dd k}{2\pi}\sum_{\omega_\nu}\frac{\p_i(k,i\omega_\nu)}{i\omega_\nu \mp v_Fk}\e^{i(kx-\omega_\nu\tau)}
\end{equation}
with upper ($i=1$) and lower($i=2$). We have now in principle found
\begin{equation}
	\gre_i(x,\tau;x',\tau',\p_i) = \gre_i^0(x-x',\tau-\tau')\e^{-i\left[f_i(x,\tau,\p_i)-f_i(x',\tau',\p_i)\right]},
\end{equation}
which was the ansatz. But this is \underline{not} the physical electron-propagators for the system! 
The propagator \(\gre_i\) seperates from the free propagators only by a phase factor, and must be understood as free fermion propagators in a given bosonic background field where the field configuration for the bosons are specified.
The physical fermion propagator is found by averaging $\gre_i$ over the fields $\p_i$. 
We thus want to compute
\begin{equation}
\label{eq:averaged_greens}
\ev{\gre_i}_{\p_i} = \frac{\int\D\p^*\D\p\gre_i(x, \tau;x', \tau', \p_i)\e^{\Sa_{\text{eff}}}}{\int\D\p^*\D\p\e^{\Sa_{\text{eff}}}}.
\end{equation}
To get any further with this problem, we must have an explicit expression for $\Sa_{\text{eff}}$ in terms of $\p^*,\p$. We will look more into this, but the followin discussion will be general to begin with. We had \cref{eq:s_eff}
\begin{equation*}
\Sa_{\text{eff}} = \int_{0}^{\beta}\dd\tau\int_{0}^{L}\dx\frac{\p^*\p}{g} + \Tr\ln(D_1+i\p_1) + \Tr\ln(D_2 +i\p_2),
\end{equation*}
which, for a free system is
\begin{equation}
\Sa_0 =\sum_{i=1}^2\Tr\ln(D_i).
\end{equation}
This contribution is of no importance when averaging over $\p_i$-configurations, and we therefore substract it.
Remember:
\begin{align*}
	(D_i + i\p_i) &= -\gre_i^{-1}(x,\tau;x',\tau',\p_i) \\
	(D_i + i\p_i)&^{-1} = -\gre_i(x,\tau;x',\tau',\p_i)
\end{align*}

\begin{align*}
\ln(D_i + i\p_i) &= \ln D_i + \ln\left(\frac{D_i+i\p_i}{D_i}\right) \\
&=\ln D_i +i\p_i\int_0^1\dd\lambda\left(D_i+\i\p_i\lambda\right)^{-1} \\
&=\ln D_i - i\p_i\int_0^1\dd\lambda\underbrace{\gre_i(x,\tau;x'\tau',\lambda\p_i)}_{\text{matrix}}
\end{align*}
$\Tr\ln(D_i+i\p_i)$ means suming over the diagonal elements of $\ln(D_i+i\p_i)$. The first term gives a free-electron contribution, and is known. 
We now have to calculate the sum of the diagonal-elements in the matrix
\begin{equation}
	\p_i(x,\tau)\int_{0}^{1}\dd\lambda\gre_i(x,\tau;x'\tau',\lambda\p_i),
\end{equation}
which involves letting $x\rightarrow x', \tau\rightarrow\tau'$ in $\gre_i$ and thereafter integrate over $\lambda$.
Since $\gre_i^0(x-x',\tau-\tau')$ is divergent in this limit, we must be careful when considering this limit. First look at $\gre_1$:
\begin{equation}
\gre_1 = \gre_1^0\e^{-i\left[f_1(x,\tau)-f_1(x',\tau')\right]}.
\end{equation}
Set $\tau =\tau'$. Now we can evaluate the limit $x\rightarrow x'$ in two ways; $x'$ can approach $x$ either from above or below \[x'\rightarrow x\pm\eta;\quad \eta = 0^+\]
We treat the transition as an average of both ways.
\begin{align}
\gre_1^0 &= \frac{1}{2\pi}\frac{\e^{ik_F(x-x')}}{i(x-x') \mp ik_F\eta} \\
&\implies \frac{1}{2\pi}\frac{\e^{\mp ik_F\eta}}{i\mp\eta}\rightarrow \mp\frac{1}{2\pi\eta i}
\end{align}
%\subsection{The phase factor} % 258 / 203
Consider now the phase factor of $\gre_1$;


\begin{align*}
f_1(x,\tau) - f_1(x\pm\eta,\tau) &= f_1(x,\tau) - f_1(x,\tau) \mp\eta\pdv{f_1}{x} \\
&= \mp\eta\pdv{f_1}{x} \\
\lim\limits_{x\rightarrow x'}\gre_1(x,\tau;x',\tau,\lambda\p_1) &= \frac{1}{2}\left(\frac{-1}{2\pi i}\right)\left[\frac{1}{\eta}\e^{i\eta\pdv{f}{x}} - \frac{1}{\eta}\e^{-i\eta\pdv{f}{x}}\right] \\
&= \frac{-1}{4\pi i\eta}\left[1+i\eta\pdv{f_1}{x} - \left(1-i\eta\pdv{f_1}{x}\right)\right] \\
&= -\frac{1}{2\pi}\pdv{f_1(x,\tau,\lambda\p_1)}{x} \\
\lim\limits_{x\rightarrow x'}\gre_2(x,\tau;x',\tau,\lambda\p_2) &= \frac{1}{2\pi}\pdv{f_1(x,\tau,\lambda\p_2)}{x}
\end{align*}
\begin{align*}
\Sa_{\text{eff}} = \Sa_0 &- \int_0^L\dx\int_0^\beta\dd\tau\frac{\p_1\p_2}{g}\\&-i\int_0^1\dd\lambda\int_0^L\dx\int_0^\beta\dd\tau\left[\p_1\left(\frac{-1}{2\pi}\pdv{f_1(\lambda\p_1)}{x}\right) + \p_2\frac{1}{2\pi}\pdv{f_2(\lambda\p_2)}{x}\right] \\
=\Sa_0 &- \int_0^L\dx\int_0^\beta\dd\tau\frac{\p_1\p_2}{g}\\
&+\frac{i}{2\pi}\underbrace{\int_0^1\dd\lambda\,\lambda}_{\flatfrac{1}2}\int_0^L\dx\int_0^\beta\left[\p_1\pdv{f_1(\p_1)}{x} -\p_2\pdv{f_2(\p_2)}{x}\right],
\end{align*}
where we used that \(f_i(x, \tau, \lambda\p_io) = \lambda f_i(x, \tau, \p_i)\) to extract and integrate out $\lambda$.
But \(f_i\sim\p_i\) implies that \( \Sa_{\text{eff}}(\p^*,\p)\) is quadratic in the $\p$-fields. Thus, we have an exact effective action as a free boson theory!
It is worth noticing how this remarkable result was achieved; It originates from the form \[\gre_i=\gre_i^0\e^{-i(f_i-f_i')}\]
Naturally, we can write an ansatz like this for the Green's function for an arbitrary, if only the $f_i$-fields are properly chosen. In our case, where the spectrum is \textit{linearized}, the differential equation for $f_i$ is especially simple, such that $f_i \sim\p_i$. Because of this, $\Sa_{\text{eff}}$ becomes quadratic in $\p_i$. More generally, $f_i$ would become complicated functions of $\p_i$, and the resulting boson-theory would be complicated and interacting. 

In the following we will exclude \(\Sa_{0}\) from \(\Sa_{\text{eff}}\) as it does not contribute to the averaging over the $\p_i$-fields. The effective action is
\begin{equation}
\label{eq:1d_effective_action}
\Sa_{\text{eff}} = -\int_0^L\dx\int_0^\beta\dd{\tau}\left[\frac{\p_1\p_2}{g} - \frac{i}{4\pi}\left(\p_1\pdv{f_1}{x}-\p_2\pdv{f_2}{x}\right)\right].
\end{equation}

Recall the fourier transformed fields in \cref{eq:fourier_transformed_1d_fields,eq:fourier_transformed_field_2}, and insert them  in the first term of \cref{eq:1d_effective_action}. We then get
\begin{align*}
-\int\dx\int\dd\tau\frac{\p_1\p_2}{g} &= -\frac{1}{g}\int\dx\int\dd{\tau}\frac{1}{\beta}\sum_{\omega_{\nu_1}}\int\frac{\dd{k_1}}{2\pi}\frac{1}{\beta}\sum_{\omega_{\nu_2}}\int\frac{\dd{k_2}}{2\pi} \\
&\quad\times \e^{i(k_1+k_2)x - i(\omega_{\nu_1}+\omega_{\nu_2}\tau)}\p_1(k_1,\omega_{\nu_1})\p_2(k_2, \omega_{\nu_2})\\
&= -\frac{1}{\beta g}\sum_{\omega_\nu}\int\frac{\dd{k}}{2\pi}\p_1(k,\omega_\nu)\p_2(-k,-\omega_\nu),
\end{align*}
where the integration over $x, \tau$ has given $\delta$-functions in $k,\omega_\nu$.
Now take the second term, the term with $p_1$, insert the Fourier transformed forms, and get
\begin{align*}
\frac{i}{4\pi}\int\dx\int\dd{\tau}\p_1\pdv{f_1}{x} &= \frac{i}{4\pi}\int\dx\int\dd{\tau}\frac{1}{\beta}\sum_{\omega_{\nu_1}}\int\frac{\dd{k_1}}{2\pi}\p_1(k_1, \omega_{\nu_1})\e^{i(k_1x - \omega_{\nu_1}\tau)} \\&\quad\times \pdv{x}\frac{-1}{\beta} \sum_{\omega_{\nu_2}}\int\frac{\dd{k_2}}{2\pi}\frac{\p_1(k_2, \omega_{\nu_2})}{i\omega_{\nu_2} - v_Fk_2}\e^{i(k_2x - \omega_{\nu_2}\tau)} \\
&= \frac{i}{4\pi}\int\dx\int\dd{\tau}\frac{1}{\beta}\sum_{\omega_{\nu_1}}\int\frac{\dd{k_1}}{2\pi}\p_1(k_1, \omega_{\nu_1})\e^{i(k_1x - \omega_{\nu_1}\tau)} \\&\quad\times \frac{-1}{\beta} \sum_{\omega_{\nu_2}}\int\frac{\dd{k_2}}{2\pi}\frac{ik_2\p_1(k_2, \omega_{\nu_2})}{i\omega_{\nu_2} - v_Fk_2}\e^{i(k_2x - \omega_{\nu_2}\tau)} \\
&= -\frac{i^2}{4\pi}\int\frac{\dd{k}}{2\pi}\frac{1}{\beta}\sum_{\omega_\nu}\frac{k}{i\omega_\nu-v_Fk}\p_1(k, \omega_\nu)\p_1(-k,-\omega_\nu) \\
&= -\frac{1}{\beta}\int\frac{\dd{k}}{2\pi}\sum_{\omega_\nu}A_1(k,\omega_\nu)\p_1(k, \omega_\nu)\p_1(-k,-\omega_\nu), 
\end{align*}
where \[A_1(k,\omega_\nu) = -\frac{1}{4\pi}\frac{k}{i\omega_\nu - v_Fk}.\]
Similarly for the $\p_2$-term;
\[A_2(k,\omega_\nu) = \frac{1}{4\pi}\frac{k}{i\omega_\nu +v_Fk}.\]
Combining it all, we have
\begin{align}
\begin{split}
\Sa_{\text{eff}} &= -\frac{1}{\beta}\sum_{\omega_\nu}\int\frac{\dd{k}}{2\pi}\left[\frac{\p_1(k,\omega_\nu)\p_2(-k,-\omega_\nu)}{g}\right. \\
&\quad+ A_1\p_1(k,\omega_\nu)\p_1(-k,-\omega_\nu)
+ A_2\p_2(k,\omega_\nu)\p_2(-k,-\omega_\nu)\left.\vphantom{\int}\right],
\end{split}
\end{align}
effective action in quadratic form!
We are now come so far that we can compute \(\ev{\gre_i(x, \tau;x', \tau', \p_i)}_{\p_i}\)
Inserting what we know in \cref{eq:averaged_greens}, with \[\gre_1^0 = \frac{1}{2\pi}\frac{\e^{ik_F(x-x')}}{i(x-x')-v_F(\tau-\tau') + \alpha} ; \quad \alpha = 0^+.\]
$\gre_1^0$ goes outside as a multiplicative factor, since it does not depend on the bosonic fields. 
\begin{align*}
-i\left[f_1(x, \tau, \p_1) - f_1(x', \tau', \p_1)\right] &= -i\frac{-1}{\beta}\int\frac{\dd{k}}{2\pi}\sum_{\omega_\nu}\frac{1}{i\omega_\nu-v_Fk}\p_1(k, \omega_\nu) \\
&\quad\cdot\left[\e^{i(kx-\omega_\nu\tau)} - \e^{i(kx'-\omega_\nu\tau')}\right] \\
&=\frac{-i}{\beta}\sum_{\omega_\nu}\int\frac{\dd{k}}{2\pi}\p_1(k, \omega_\nu)\J_1(k, \omega_\nu, x, \tau;x', \tau'),\\
\J_1&\equiv \frac{-1}{i\omega_\nu-v_Fk} \left[\e^{i(kx-\omega_\nu\tau)} - \e^{i(kx'-\omega_\nu\tau')}\right]
\end{align*}
Now define 
\begin{equation}
\Z_\p = \int\D\p_1\D\p_2\e^{\Sa_{\text{eff}}}.
\end{equation}
The average can be written as
\begin{align*}
\ev{\gre_1}_\p = \frac{\gre_1^0}{\Z_\p} \int \D\p_2\D\p_2 &\exp \left\{ -\frac{1}{\beta}\sum_{\omega_\nu}\int\frac{\dd{k}}{2\pi} \left[\frac{\p_1\p_2}{g}\right.\right.\\
& \left.\left.+A_1\p_1\p_1+ A_2\p_2\p_2+ \frac{i}{2}\left(\p_1\J_1 + \p_2\J_2\right)\right]\vphantom{-\frac{1}{\beta}\sum_{\omega_\nu}\int\frac{\dd{k}}{2\pi} \left[\frac{\p_1\p_2}{g}\right. }\right\}.
\end{align*}
\underline{The exponent:}

\begin{align*}
\p_1(k)\frac{1}{g}\p_2(-k) &+ A_1\p_1(k)\p_1(-k) \\
&\quad + A_2\p_2(k)\p_2(-k) + \frac i2\left(\p_1(k)\J_1(k) + \underbrace{\p_1^*(-k)}_{\sim\p_2(-k)}\J_1^*\right) \\
&= \left(\p_1+\frac{i}{2}D\J_1^*\right)D^{-1}\left(\p_2+\frac{i}{2}D\J_1\right) + \frac{D}{4}|\J_1|^2. 
\end{align*}
This allows us to write
\begin{equation}
\label{eq:gre_almost_final}
\ev{\gre_1}_\p = \gre_1^0\frac{1}{\Z_\p}\Z_\p\e^{-Q(x, \tau;x',\tau')}
\end{equation}
with
\begin{equation}
\label{eq:Q}
Q = \frac{1}{4\beta}\sum_{\omega_\nu}\int\frac{\dd{k}}{2\pi}D|\J_1|^1, .
\end{equation}
and
\begin{align*}
|\J_1|^2 &= 2\left[1-\cos(k(x-x')-\omega_\nu(\tau-\tau'))\right]\frac{1}{\omega_\nu^2 + k^2v_F^2}\\
D^{-1} &= \frac{1}{g} + A_1+A_2 \\
&= \frac1g-\frac{k}{4\pi}\left(\frac{1}{i\omega_\nu - v_Fk}-\frac{1}{i\omega_\nu+v_Fk}\right) \\
&=\frac{1}{g}-\frac{k}{4\pi}\frac{2v_Fk}{(i\omega_\nu)^2-(v_Fk)^2} \\
&=\frac{1}{g}\frac{(i\omega_\nu)^2-(v_Fk)^2-(v_Fk)^22\tilde{g}}{(i\omega_\nu)^2-(v_Fk)^2} \\
&= \frac1g \frac{(i\omega_\nu)^2-(v_Fk)^2(1+2\tilde{g}}{(i\omega_\nu)^2-(v_Fk)^2},
\end{align*}
where we defined \(\tilde{g} = \frac{g}{4\pi v_F}\). Multiplying \(D\) with \(|\J|^2\), we obtain
\begin{equation}
D|\J_1|^2 = \frac{2g\left(1-\cos\left[k(x-x')-\omega_\nu(\tau-\tau')\right]\right)}{\omega_\nu^2+(v_Fk)^2(1+2\tilde{g})}.
\end{equation}
Inserting this into \cref{eq:Q}, we get
\begin{equation}
Q(x, \tau) = \frac{g}{2\beta}\sum_{\omega_\nu}\int\frac{\dd{k}}{2\pi}\frac{\left[1-\cos\left(kx-\omega_\nu\tau\right)\right]}{\omega_\nu^2+(v_Fk)^2(1+2\tilde{g})}
\end{equation}
Set \(\tau = 0\) and do the frequency summation
\begin{equation*}
	\frac1\beta\sum_{\omega_\nu}\frac{1}{\omega_\nu^2 +\varepsilon_k^2} = -\frac{1}{2\varepsilon_k}\left(\frac{1}{\e^{-\beta\varepsilon_k}-1} - \frac{1}{\e^{\beta\varepsilon_k}-1}\right)
\end{equation*}
At $T = 0$, the first term only contributes if $k>0$ and the second only if $k<0$. 
\begin{align*}
	Q(x) &= -\frac{g}{2}\frac{1}{2\pi v_F\sqrt{1+2\tilde{g}}}\left[\int_{-\infty}^{0}\dd{k}\frac{1-\cos(kx)}{k}-\int_0^\infty\dd k\frac{1-\cos(kx)}{k}\right] \\
	&=\underbrace{\frac{g}{2}\frac{2}{2\pi v_F\sqrt{1+2\tilde{g}}}}_{\equiv 2\nu}\underbrace{\int_0^\infty\dd k\left(\frac{1-\cos(kx)}{k}\right)\e^{-\Lambda k}}_{\equiv H(x)}.
\end{align*}
Thus we have, with \(\nu = \frac{\tilde{g}}{\sqrt{1+2\tilde{g}}}\),
\begin{equation}
Q(x) = 2\nu H(x)
\end{equation}
The factor \(e^{-\Lambda x}\) is introduced as a cutoff on the momentum transfer in a forward-scattering process. This \(\Lambda\)-factor is \underline{not} related to the facto \(\e^{-\alpha|k|}\) which we used as a cutoff on the spectrum for the free fermions. NB: Notice how completely wrong a mean field theory of the bosonic sector would be, in this case!
\begin{align*}
\pdv{H}{x} &= \int_0^\infty\dd k \e^{-\Lambda k}\sin(kx) \\
&= \frac{x}{x^2+\Lambda^2} \\
\implies H &= \frac12\ln\left(\frac{x^2+\Lambda^2}{\Lambda^2}\right) \\
\implies Q(x) &= 2\nu\frac12\ln\left(\frac{x^2+\Lambda^2}{\Lambda^2}\right) \\
&= \ln\left[\left(\frac{x^2+\Lambda^2}{\Lambda^2}\right)^\nu\right] 
\end{align*}
We can now write down the Green's function by inserting the results in \cref{eq:gre_almost_final} and obtain
\begin{equation}
\ev{\gre_1}_\p = -\ev{\psi_1\psi_1^\dagger} = \frac{1}{2\pi}\frac{\e^{ik_F(x-x')}}{i(x-x')+\alpha}\cdot\left(\frac{\Lambda^2}{(x-x')^2+\Lambda^2}\right)^\nu
\end{equation}
Notice how the propagator returns to a free fermion propagator if the interaction is zero (\(g\rightarrow0\)). 
The propagator has now gained a singularity that is not only simple poles. No longer a Fermi-fluid!
The factor \(\left(\frac{\Lambda^2}{x^2+\Lambda^2}\right)\) describes a ``cloud'' of particle-hole excitations that the electron ``carries'' along.
\subsection{The bosonic excitation spectrum} % 270 / 212a
\subsection{Impulse distribution} % 277 / 212f  Kanskje ta side 212a - 212g i samme subsection?
\subsection{``Summary''} % 279 / 213
