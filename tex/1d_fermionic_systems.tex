\section{Interacting one-dimensional fermionic systems}
%(page 233 pdf, 182 notes)
%\subsection{Intro} % Snorre

\subsection[Model]{1D interacting model}

We model the system by $\Ha = H_0 + H_I$ as usual. Consider first $H_0$
\begin{align*}
	\Ha_0 &= \sum_{i,j,\sigma}t_{ij}\cd_{i\sigma}c_{j\sigma} \\
	&\implies \sum_{k \sigma}\varepsilon_k\cd_{k\sigma}c_{k\sigma}
\end{align*}

\begin{figure}
	\centering
	\begin{tikzpicture}
	\begin{axis}[
	legend style={at={(1,0.4)}},
	ytick = {1},
	ytick style={draw=none},
	yticklabels = {},
	xtick = {-3.14159,3.14159},
	xticklabels ={$-\pi$,$\pi$},
	xlabel = \large $k$,
	ylabel = \large $\varepsilon_k$,
	axis lines = center,
	ymin = -1.5,
	ymax = 1.5,
	xmax = 4,
	xmin = -4,
	%x label style={at={(axis description cs:1,0.1)}, anchor = west},
	%y label style={at={(axis description cs:0.15,1)},rotate=-90,anchor=south},
	%xlabel = $x$,
	%sylabel = {$f(x)$},
	]
	
		
		\addplot[name path = A,domain = -1.57:1.57, samples = 200, very thin, black]{(-cos(deg(x)))};
		\addplot[domain = -3.14:3.14, samples = 200, thick, red]{(-cos(deg(x)))};
		\addplot[name path = B, domain = -1.57:1.57] {0*x};
		
		\addplot[blue!10!white] fill between[of = A and B];
		\draw[thick, black, dashed] (axis cs:-1.57,0) -- node[anchor= north east]{\large $-k_F$}  (axis cs:-1.570,-0.5);
		\draw[thick, black, dashed] (axis cs:1.57,0) -- node[anchor= north west]{\large $k_F$}  (axis cs:1.570,-0.5);
	\end{axis}
\end{tikzpicture}
	\caption{Dispersion relation. The fermi surface consists of two points}
	\label{fig:disp_rel}
\end{figure}

We are mainly interested in the low energy physics for this problem; We want to describe $\varepsilon_k$ in the vicinity of the Fermi level. In this case, it is especially simple as it only consists of two points, as seen in Figure \ref{fig:disp_rel}.

Right-moving: $k>0$ and left-moving: $k<0$. For both of which we have  $\varepsilon_k\simeq V_F(|k| - k_F)$ and $|k-k_F|<<k_F$. $\varepsilon_k$ is here measured relative the Fermi level.
For every electron spin $\sigma$ we have one type of righ-moving and one type of left-moving fermions, seen in Figure \ref{fig:lr_fermions}.

\begin{figure}
	\centering
	\input{tex/img/rl_fermions}
	\caption{The right- and left-moving fermions of interest.}
	\label{fig:lr_fermions}
\end{figure}

\begin{align}
\label{eq:h0_rl}
\begin{split}
\Ha_0 &\simeq \sum_{k,\sigma}V_F(k-k_F)\cd_{1k\sigma}c_{1k\sigma} \\
&+\sum_{k,\sigma}V_F(-k-k_F)\cd_{2k\sigma}c_{2k\sigma},
\end{split}
\end{align}
where $\cd_{1k\sigma}$ creates a right-moving fermion and $\cd_{2k\sigma}$ creates a left-moving fermion etc.
NB: \ref{eq:h0_rl} is defined only at $|\pm k-k_F|<<k_F$!
We write this as
\begin{align}
\begin{split}
\Ha=\sum_{k,\sigma}&\left[\e^{-\alpha|k-k_F|}V_F(k-k_F)\cd_{1k\sigma}c_{1k\sigma}\right. \\
&+\left.\e^{-\alpha|k-k_F|}V_F(-k-k_F)\cd_{2k\sigma}c_{2k\sigma}\right].
\end{split}
\end{align}
Here, $\alpha>0$ is a cutoff-parameter to limit the $k$-summation to those who are close to Fermi level, but in such a way that we can sum over all $k$.
Introduce the real space fermion operators
\begin{align}
\psi_{1\sigma}(x) & =\int\frac{\dd k}{2\pi}c_{1k\sigma}\e^{ikx} \\
\psi_{2\sigma}(x) & =\int\frac{\dd k}{2\pi}c_{1k\sigma}\e^{-ikx},
\end{align}
where the sums over $k$ is calculated close to Fermi-level.
Consider now the free fermion propagators in real space at equal times $(\tau = \tau')$

\begin{align*}
\mathcal{G}_1(x) &= -i\ev{:\psi_{1\sigma}(x)\psi_{1\sigma}^\dagger(x'):}\\
&= -i\int\frac{\dd k}{2\pi}\int\frac{\dd k'}{2\pi}\e^{ik(x-x')}\underbrace{\ev{:c_{1k\sigma}\cd_{1k\sigma}:}}_{=-\delta_{k,k'}n_k} \\
&=i\int_{k<k_F}\frac{\dd k}{2\pi}\e^{-\alpha|k-k_F|}\e^{ik(x-x')} \\
&= i\int_{q<0}\frac{\dd q}{2\pi}\e^{-\alpha|q|}\e^{i(q+k_F)(x-x')} \\
&=i\e^{ik_F(x-x')}\int_{-\infty}^{0}\frac{\dd k}{2\pi}\e^{-\alpha|k| + ik(x-x')}\\
&= i\frac{\e^{ik_F(x-x')}}{2\pi}\int_0^\infty\dd k\,\e^{-k(\alpha + i(x-x'))} \\
&=i\frac{\e^{ik_F(x-x')}}{2\pi}\frac{1}{\alpha + i(x-x')},
\end{align*}
or, to summarize, as $\alpha \rightarrow 0$ (ignoring the short-range cutoff)
\begin{align}
\mathcal{G}_1 &= \frac{1}{2\pi}\frac{\e^{ik_F(x-x')}}{x-x'} \\
\mathcal{G}_2 &= -\frac{1}{2\pi}\frac{\e^{-ik_F(x-x')}}{x-x'}.
\end{align}
Notice the simple poles in the Green's functions. Now, by the $\psi_{1\sigma}$ and $\psi_{2\sigma}$-operators, we have

\begin{equation}
\Ha_0 = V_F\int_{0}^{L}\dd x\left[\psi_{1\sigma}^\dagger\left(-i\partial_x -k_F\right)\psi_{1\sigma} + \psi_{2\sigma}^\dagger\left(i\partial_x - k_F\right)\psi_{2\sigma}\right].
\end{equation}
Defining $\psi_\sigma = \begin{psmallmatrix*} \psi_{1\sigma} & \psi_{2\sigma}
\end{psmallmatrix*}^T$ lets us rewrite this as
\begin{equation}
\Ha_0 = V_F\sum_\sigma\int_{0}^{L}\dd x\,\psi_\sigma^\dagger
\underbrace{\begin{pmatrix}
-i\partial_x - k_F & 0 \\
0 &  i\partial_x - k_F 
\end{pmatrix}}_{A_0}\psi_\sigma
\end{equation}

We now add an interaction term to the theory, on the form
\begin{align*}
\Ha_I &= \sum_{\sigma,\sigma'}\int\dd x\int\dd y\underbrace{\psi_{1\sigma}^\dagger(x)\psi_{1\sigma}(x)}_{\rho_{1\sigma}(x)}v(x-y)\underbrace{\psi_{2\sigma}^\dagger(y)\psi_{2\sigma'}(y)}_{\rho_{2\sigma'}(y)} \\
&= \int \dx\dy \rho_1(x)v(x-y)\rho_2(y).
\end{align*}
We do not consider terms of the types $\rho_1\rho_1, \rho_2\rho_2$; these only contributes to the renormalization of $V_F$. $\rho_1\rho2$ will, however, give rise to more intersting effects.

Therefore, we want to study the theory of

\begin{align}
\begin{split}
\Ha &= V_F \sum_\sigma \int\dx\psi_\sigma^\dagger A_0\psi_\sigma \\
&+ \sum_{\sigma,\sigma'}\int \dx\dy \rho_{1\sigma}(x)v(x-y)\rho_{2\sigma'}(y).
\end{split}
\end{align}

For simplicity, we will in the continuation consider spinless fermions. A spin treatment will be given later. 

\begin{equation}
\Ha = V_F \int\dx\psi^\dagger A_0\psi+ \int \dx\dy \rho_{1}(x)v(x-y)\rho_{2}(y).
\end{equation}
A spinless system is the same (or equivalent) to a completely spin-polarized problem. We can write down the partition function by applying our normal rules;
\begin{align}
\Z &= \int\D\psi_1^\dagger\D\psi_1\D\psi_2^\dagger\D\psi_2\e^{\Sa} \\
\begin{split}
\Sa &= -\int\dx \int_0^\beta\dd\tau\left(\psi^\dagger\partial_\tau\psi + V_F\psi^\dagger A_0\psi\right) \\
&- \int\dx\int\dy \int_0^\beta\dd\tau\rho_{1}(x)v(x-y)\rho_{2}(y)
\end{split}
\end{align}
Let us simplify the model some more by only considering constant \textit{contact}-interaction between fermions 
\begin{equation}
	v(x-y) = g\delta(x-y); \quad g>0.
\end{equation}

\begin{align*}
	\Sa &= \Sa_0 + \Sa_I \\
	\Sa_0 &= -\int_0^L\dx\int_0^\beta\dd\tau\psi^\dagger\left(\partial_\tau+ V_FA_0\right)\psi \\
	\Z_0 &= \e^{\Tr\ln(\partial_\tau + V_FA_0)} \\
	&= \e^{\int_0^L\dx\int_0^\beta\dd\tau\tr\ln\left(\partial_\tau+ V_FA_0\right)} \\
	\Sa_I &= -g\int\dx\int\dd\tau\rho_1(x)\rho_2(x)
\end{align*}
Now we perform a Hubbard-Stratonovich decoupling og the interaction term in the following way;
\begin{align*}
\int\D\pdag\D\p\,\e^{-(\pdag + g\rho_2)(\p + g\rho_1)g^{-1}} &= \text{const} \\
&= \e^{-\Tr\ln(g^{-1})} = \e^{\ln g} = g.
\end{align*}
Correspondingly,we have
\begin{align*}
\int\D\pdag\D\p\,&\e^{-\int\dx\dd \tau\left(g^{-1}\p^*\p+i\p^*\rho_1 + i\p^*\rho_2\right)} \\
&= \int\D\pdag\D\p\,\e^{-\int\dx\dd \tau\left[g^{-1}(\p^* + ig\rho_2)(\p + ig\rho_1) + g\rho_1\rho_2\right]} \\
&= \e^{-\int\dx\dd \tau g\rho_1\rho_2}\cdot \text{const}.
\end{align*}
The constant is irrelevant and is safely ignored.
\begin{align}
\e^{\Sa_I} &= \int\D\p^*\D\p\e^{-\int\dx\int_0^\beta\left(g^{-1}\p^*\p+i\p^*\rho_1 + i\p\rho_2\right)} \\
\Z &= \int\D\p^*\D\p\D\psi_1^\dagger\D\psi_1\D\psi_2^\dagger\D\psi_2\e^{\Sa[\psi^\dagger,\psi,\p^*\p]}
\end{align}
\begin{equation}
\label{eq:action_1d}
\Sa= -\int_0^L\dx \int_0^\beta \dd\tau \left\{\frac{\p^*\p}{g} +  \psi^\dagger A\psi\right\},
\end{equation}
where\footnote{There is some ambiguity here. In the notes, $\vb{I}$ is denoted $\vec{T}$, so it should be checked that it is the identity operator. We should have a term with $\psi^\dagger\partial_\tau\psi$, so the identity makes sense.}
\begin{equation}
	A \equiv V_FA_0 + \vb{I}\partial_\tau + i
	\begin{pmatrix}
	\p^* & 0 \\
	0 & \p
	\end{pmatrix} = -\mathcal{G}^{-1}
\end{equation}
and
\begin{equation}
	\mathcal{G}(\partial_\tau, x) = -\ev{\psi\psi^\dagger}.
\end{equation}

Note that there is no terms $\sim \p^*\partial_\tau\p$ in \eqref{eq:action_1d}, because the $\p$-field is absent in the Hamiltonian, but is used to decouple the interaction first after the partition function is written down. If we use the spinor-components reduces to 
\begin{equation}
\mathcal{G}(\partial_\tau,x) = -\begin{pmatrix}
\ev{\psi_1\psi_1^\dagger} & 0\\
0 & \ev{\psi_2\psi_2^\dagger}
\end{pmatrix}.
\end{equation}
Since the matrix is diagonal, we can immidiately write down 
\begin{align}\label{eq:greeenss}
\begin{split}
\mathcal{G}_1 &= -(-iV_F\partial_x + \partial_\tau - V_Fk_F + i\p^*)^{-1} \\
\mathcal{G}_2 &= -(iV_F\partial_x + \partial_\tau - V_Fk_F + i\p)^{-1}
\end{split}
\end{align}
The Greens functions \eqref{eq:greeenss} are free Greens functions in \underline{fixed} bosonic background field.


\subsection[Free propagators]{Free fermion propagators} % pdf: 245 / notes: 192
\subsection[Interacting propagator]{Propagators in  the interacting case} % pdf:248 / notes 196
\subsection{The phase factor} % 258 / 203
\subsection{The bosonic excitation spectrum} % 270 / 212a
\subsection{Impulse distribution} % 277 / 212f  Kanskje ta side 212a - 212g i samme subsection?
\subsection{``Summary''} % 279 / 213
