\section[Examples]{Two important examples of the formalism}

\subsection{Free electron gas}
\label{sec:free_electron}

We start with the Hamiltonian
\begin{align}
\Ha &= \sum_{k,\sigma}\varepsilon_kc_{k\sigma}^\dagger c_{k\sigma}\nonumber \\
& = \sum_\sigma \int \dx \psi_\sigma^\dagger(x)\varepsilon(\nabla)\psi_\sigma(x).
\end{align}
The partition function is 

\begin{equation}
\label{eq:partition_integral}
\Z = \int \mathcal{D}\left[\varphi^*(\tau)\right]\mathcal{D}\left[\varphi(\tau)\right]\e^{\mathcal{S}}
\end{equation}
where \( \varphi_\lambda(0) = -\varphi_\lambda(\beta)\) (antiperiodic for fermions) and
\begin{equation}
\mathcal{S} = -\sum_\lambda\int_0^\beta\dd{\tau}\left[\varphi_\lambda^*\pdv{\varphi_\lambda}{\tau} + \Ha \left(\{\varphi_\lambda^*, \varphi_\lambda\}\right)\right]
\end{equation}
Now choose quantum numbers \(\lambda = \left(k,\sigma\right)\) because $\Ha$ is diagonal in the plane wave basis. Then,  
\begin{equation}
\mathcal{S} = -\sum_{k,\sigma}\int_0^\beta\dd{\tau}\varphi_{k\sigma}^*(\tau)\left(\pdv{\tau} + \varepsilon_k\right)\varphi_{k\sigma}(\tau)
\end{equation}
where $\{\varphi_{k\sigma}(\tau)\}$ are Grassman variables. $\Z$ now becomes a Gaussian integral over Grassmann variables, which we have seen earlier. By direct insertion of this result, we find 

\begin{align}
\Z &= \e^{\Tr \ln\left(\partial_\tau + \varepsilon_k\right)} \label{eq:partition_free_electron1}\\
&\stackrel{?}{=} \prod_{k,\sigma}\left(1+\e^{-\beta\varepsilon_k}\right)\nonumber
\end{align} 
with 
\begin{equation}
\label{eq:trace_}
\Tr = \sum_{k,\sigma}\int_0^\beta\dd{\tau}\cdot \tr
\end{equation}
where ``tr'' here is the trace of the operator $\ln\left(\partial_\tau + \varepsilon_k\right)$
\begin{equation}
\tr\ln\left(\partial_\tau + \varepsilon_k\right) = \sum_n\ev{\ln\left(\partial_\tau + \varepsilon_k\right)}{n}.
\end{equation}
To be able to get a \underline{local} expression for $\ln\left(\partial_\tau + \varepsilon_k\right)$, the choice of a plane wave basis for $\ket n$ is convenient.

\begin{equation}
\ket n = u_{nk} = \frac{1}{\sqrt{\beta}}\e^{i\left(\textbf{k}\cdot\textbf{r} - \omega_n\tau \right)}
\end{equation}
where
\begin{equation}
\omega_n = \frac{\left(2n+1\right)\pi}{\beta}.
\end{equation}
The reason for this choice of $\omega_n$ is that we see that this ensures $u_{nk}(\tau)$ to have the same \underline{antiperiodic} properties as $\varphi_\lambda(\beta)$
When we take the trace only over such states, the requirement \( \varphi_\lambda(0) = -\varphi_\lambda(\beta)\) is automatically satisfied.

\begin{align}
&\sum_n \ev{\ln\left(\partial_\tau + \varepsilon_k\right)}{n} & \nonumber \\
& = \frac{1}{\beta}\sum_{\omega_n}\e^{-i\left(\textbf{k}\cdot \textbf{r} - \omega_n\tau\right)}\ln\left(\partial_\tau + \varepsilon_k\right)\e^{i\left(\textbf{k}\cdot \textbf{r} - \omega_n\tau\right)}. \label{eq:trace_free_electron}&
\end{align}

Before we continue, we investigate the trace of an arbitrary operator
\begin{equation}
\tr\ln A = \sum_n\ev{\ln A}{n}.
\end{equation}

$\ln A$ is defined by its series expansion

\begin{align}
\ln A &= \ln (1+A-1) \nonumber \\
&= \sum_{k=1}^\infty\frac{(-1)^{k+1}}{k}\left(A-1\right)^k,
\end{align}
such that 
\begin{equation}
\tr\ln A = \sum_{k=1}^\infty\frac{(-1)^{k+1}}{k}\tr\left[\left(A-1\right)^k\right].
\end{equation}
Define $B = A-1$. Now choose $S$ such that $S^{-1}BS = S^{-1}AS -1 = D-1$, i.e. such that $A$ is diagonalized. 

\begin{align}
\tr(B^k) &= \tr\left[\left(D-1\right)^k\right] \nonumber\\
&= \sum_m \left(\lambda_m -1\right)^k
\implies\nonumber \\
\tr\ln A &= \sum_m\sum_k\frac{(-1)^{k+1}}{k}(\lambda_m-1)^k\nonumber \\
&= \sum_m\ln(1+\lambda_m -1)\nonumber \\
&= \sum_m \ln\lambda_m = \ln \left(\prod_m\lambda_m\right)\nonumber \\
\implies & \tr\ln A = \ln\det A. \label{eq:trace_op}
\end{align}
When we use \eqref{eq:trace_op} in \eqref{eq:trace_free_electron}, we get

\begin{equation}
\sum_n\ev{\ln\left(\partial_\tau + \varepsilon_k\right)}{n} = \frac{1}{\beta}\sum_{\omega_n}\ln\left(-i\omega_n + \varepsilon_k\right)
\end{equation}

\begin{align}
\label{eq:_electron}
\begin{split}
\Z &= \e^{\sum_{k,\sigma}\frac{1}{\beta}\int_0^\beta\dd{\tau}\sum_{\omega_n}\ln\left(-i\omega_n + \varepsilon_k\right)} \\
&= \e^{\sum_{k,\sigma}\sum_{\omega_n}\ln\left(-i\omega_n + \varepsilon_k\right)} \\
&= \prod_{k,\sigma}\e^{\sum_{\omega_n}\ln\left(-i\omega_n + \varepsilon_k\right)}.
\end{split}
\end{align}
To get any further, we need to execute the summation over the Matsubara frequencies $\omega_n$. To do this, observe that $i\omega_n$ are the poles of the Fermi distribution
\begin{equation}
\label{eq:fermi_dist}
f(z) = \frac{1}{1+\e^{\beta z}}
\end{equation}
If a complex valued function $g(z)$ defined on $\mathbb{C}$ has a simple pole at $z = z_0$, Cauchy's residue theorem tells us that 
\begin{align}
\oint\dd{z}g(z) &= 2\pi i\Res\left[g(z_0)\right] \\
\Res\left[g(z_0)\right]&= \lim_{z\rightarrow z_0}\left[(z-z_0)g(z)\right]
\end{align}

So for the Fermi distribution in \eqref{eq:fermi_dist}, we get

\begin{align}
\Res\left[f(i\omega_n)\right] &= \lim_{z\rightarrow i\omega_n}\left[(z-i\omega_n)f(z)\right] \nonumber \\
&= \lim_{z\rightarrow i\omega_n}\frac{z-i\omega_n}{1+\e^{\beta(z-i\omega_n + i\omega_n)}} \nonumber \\
&= \lim_{z\rightarrow i\omega_n} \frac{z-i\omega_n}{1-\e^{\beta(z-i\omega_n)}} \nonumber \\
&=\lim_{z\rightarrow i\omega_n} \frac{z-i\omega_n}{1-1-\beta(z-i\omega_n)+\dots} \nonumber \\
&=\lim_{z\rightarrow i\omega_n} -\frac{1}{\beta}\frac{z-i\omega_n}{z-i\omega_n} \implies \nonumber \\
\Res\left[f(i\omega_n)\right] &= -\frac{1}{\beta}
\end{align}

We then have

\begin{align}
\oint \dd{z}f(z) &= 2\pi i \Res f(z_0) \\ 
&= -\frac{2\pi i}{\beta} \implies \\
\sum_{\substack{i\omega_n \\ \omega_n \text{odd}}} g(i\omega_n) &= -\frac{\beta}{2\pi i}\oint \dd{z}g(z)f(z)\equiv I
\end{align}
where the path encloses \underline{all} simple poles of the Fermi distribution \eqref{eq:fermi_dist} and 
\begin{equation}
g(i\omega_n)=\ln(-i\omega_n + \varepsilon_k)
\end{equation}


\begin{figure}
	\centering
	\begin{tikzpicture}[scale = 3]
\draw [-, thick] (0, 1) to (2, 1) ;

\draw [-, thick] (1, 0.1) to (1,1.9) ;

\foreach \y in {0.1,0.3, 0.5, 0.7, 0.9, 1.1, 1.3, 1.5, 1.7, 1.9}
	\node at (1, \y) {$\cross$};

\draw[red,thick,dashed,   ->] (1.2, 0.6) to [in = 0,out = 90] (1, 2) to [in = 90, out = 180] (0.8, 1.4);

\draw[red,thick, dashed, ->] (0.8, 1.4) to [in = 180,out = 270] (1, 0) to [in = 270, out = 0] (1.2, 0.6);

\node at (1.4, 1) (e){$\cross$};
\draw[] (e) node[anchor = north west] {$\varepsilon_k$};

\node[anchor = west] at (1.2, 0.4) {$\sim C$};
\end{tikzpicture}
\end{figure}

\begin{figure}
	\centering
	\label{fig:path_deform}
	\begin{subfigure}{0.49\textwidth}
		\begin{tikzpicture}[scale = 3]

  
\draw [-, thick] (0, 1) to (2, 1) ;

\draw [-, thick] (1, 0) to (1,2) ;



\foreach \y in {0.8, 1.2, 0.4, 1.6}
	\node at (1, \y) {$\cross$};


\draw[red, dashed, ->] (0.7, 1.8) parabola bend (1,1.05) (1.3, 1.8);
	
\draw[red, dashed, ->] (1.3, 0.2) parabola bend (1,0.95) (0.7, 0.2);	

\end{tikzpicture}
		
	\end{subfigure}
	\begin{subfigure}{0.49\textwidth}
		\begin{tikzpicture}[scale = 3]
\draw [-, thick] (0, 1) to (2, 1) ;

\draw [-, thick] (1, 0) to (1,2) ;

\node at (1.4, 1) (e){$\cross$};

\draw[red, dashed] (0,1.05) to (1.3, 1.05);

\draw[red, dashed, -] (1.3, 1.05) to [in=90, out = 90] (1.5, 1.05);

\draw[red, dashed, ->] (1.5, 1.05) to (2,1.05);

\draw[red, dashed] (2, 0.95) to (1.5, 0.95);


\draw[red, dashed] (1.5, 0.95) to [in = 270, out = 270] (1.3, 0.95);

\draw[red, dashed,->] (1.3,0.95) to (0,0.95);
\end{tikzpicture}
		
	\end{subfigure}
\end{figure}

Deform the path $C$ in a way that does not enclose new poles. We have to avoid the pole in $g(i\omega_n) = \ln(-i\omega_n + \varepsilon_k)$.

Consider 
\begin{align}
\tilde{I} = \frac{\beta}{2\pi i}\int_{-\infty}^\infty\dd{\varepsilon}&\left[f(\varepsilon + i\delta)\ln\left(-\varepsilon -i\delta + \varepsilon_k\right) \right.\nonumber\\
&\left.-f(\varepsilon-i\delta)\ln\left(-\varepsilon  + i\delta + \varepsilon_k\right) \right].
\end{align}

\footnote{I found no better placement as it stands on a separate page in the notes. The contribution from the pole is
	\begin{equation*}
	= -\frac{1}{2\pi i}\int_0^{2\pi}\dd{\theta}R\ln\left(R\e^{i\theta}\right) = -\frac{1}{2\pi i} R\left[2\pi\ln R + i\frac{4\pi^2}{2}\right]
	\stackrel{R \rightarrow 0}{\rightarrow}  0,
	\end{equation*}
	so no contribution.
}

This is equal to 
\begin{equation}
\label{unc:is_it_really_equal}
\tilde{I} = \frac{\beta}{2\pi i}\int_{-\infty}^\infty\dd{\varepsilon}f(\varepsilon)\left[\ln(-\varepsilon -i\delta + \varepsilon_k) - \ln(-\varepsilon + i\delta + \varepsilon_k)\right].
\end{equation}
We have to be careful, since the $\ln$-function has multiple values $\ln(z) = \ln(z) + i\varphi$, where $\varphi = 2\pi n$ for $n \in \mathbb{Z}$. We impose a branch cut off to separate the branches from one another on the Riemann surface. To eliminate the problem with a multivalued function, we define the function on specified Riemann-surfaces. The branch cut off separates one Riemann surface from another. Having multivalued functions means problems and meaninglessness when considering the computation of physical quantities. Moral of the story: \underline{Always} (properly)
examine the analytic structure of a function $g(z)$ that is included in $\sum_{\omega_n}g(i\omega_n)$.

For $\varepsilon < \varepsilon_k$, we have \(\Im(\ln z) = \pi^-\) over the real axis, and \(\Im(\ln z) = \pi^+\) under the real axis. 

\begin{align}
&\ln (-\varepsilon -i\delta + \varepsilon_k) - \ln(-\varepsilon + i\delta+\varepsilon_k) \nonumber \\
&=\ln|-\varepsilon + \varepsilon_k| + i\pi^- - \ln|-\varepsilon + \varepsilon_k| - i\pi^+\nonumber \\
&= i(\pi^- - \pi^+) = 0 \text{\footnotemark}
\end{align}
\footnotetext{According to the notes, this is not entirely correct but here the signs on $\pi$  is also swapped.}

We thus have no contribution from $\varepsilon <\varepsilon_k$!

For $\varepsilon>\varepsilon_k$, $\Im(\ln z) = 0$ over the real axis and $2\pi$ below. 

\begin{align}
&\ln (-\varepsilon -i\delta + \varepsilon_k) - \ln(-\varepsilon + i\delta+\varepsilon_k) \nonumber \\
&= \ln|-\varepsilon+ \varepsilon_k| - \ln|-\varepsilon + \varepsilon_k| + i\cdot 0 - 2\pi i = -2\pi i
\end{align}

Now we can return to the integral
\begin{align}
\tilde{I} &= -\frac{2\pi i}{2\pi i}\beta\int_{\varepsilon_k}^\infty\dd{\varepsilon}f(\varepsilon) \\
&= -\beta\int_{\varepsilon_k}^\infty\dd{\varepsilon}\frac{1}{\e^{\beta\varepsilon}+1} \\
&= -\beta\int_{\varepsilon_k}^\infty \dd{\varepsilon}\frac{\e^{-\beta\varepsilon}}{1+\e^{-\beta\varepsilon}} \\
&= \int_{\varepsilon_k}^\infty \dd{\varepsilon} \dv{\varepsilon}\ln\left( 1+\e^{-\beta\varepsilon}\right)\\
&= -\ln\left(1+\e^{-\beta\varepsilon_k}\right).
\end{align}

Thus
\begin{equation}
I = \sum_{\omega_n}\ln(-i\omega_n+\varepsilon_k) = \ln\left(1+\e^{-\beta\varepsilon_k}\right).
\end{equation}
This lets us calculate the partition function in \eqref{eq:partition_free_electron1} with the definition in \eqref{eq:trace_} as
\begin{align}
\Z &= \e^{\sum_{k\sigma}\sum_{\omega_n}\ln(-i\omega_n+\varepsilon_k)} = \e^{\sum_{k\sigma}\ln\left(1+\e^{-\beta\varepsilon_k}\right)}\nonumber \\
&= \prod_{k,\sigma}\left(1+\e^{-\beta\varepsilon_k}\right).\label{eq:fermion_partition}
\end{align}
Equation \eqref{eq:fermion_partition} is a well known result for fermions. This is the partition function for a free fermion gas with Hamiltonian 
\begin{equation}
\Ha = \sum_{k,\sigma}\varepsilon_kc_{k\sigma}^\dagger c_{k\sigma}.
\end{equation}


\subsection{Free Boson gas}

%(Page 85 in the pdf (63 in notes).)

We are now considering free, spin less bosons without any inner structure. For example phonons, magnons, solitons in one-dimensional conductors, etc..)

The Hamiltonian is 
\begin{equation}
\Ha = \sum_q \omega_qa_q^\dagger a_q
\end{equation}

As for free electron gas, we are to compute the partition function in \eqref{eq:partition_integral}, repeated here as 

\begin{equation}
\Z = \int \mathcal{D}\left[\varphi^*(\tau)\right]\mathcal{D}\left[\varphi(\tau)\right]\e^{\mathcal{S}}.
\end{equation}
This time, however, $\varphi_\lambda(0) = \varphi_\lambda(\beta)$, periodic for bosons.

\begin{equation}
\mathcal{S} = -\sum_{q}\int_0^\beta\dd{\tau}\varphi_{q}^*(\tau)\left(\partial_{\tau} + \omega_q\right)\varphi_{q}(\tau)
\end{equation}
\(\Z\) now become a multiple Gaussian integral over complex variables, since the $\varphi$'s now are eigenvalues for coherent boson states. We calculated this before;
\begin{align*}
\Z &= \e^{-\Tr\ln(\partial_\tau + \omega)} \\
\Tr &= \sum_q\int_0^\beta\dd{\tau}\tr
\end{align*}
We thus have to find a local expression for 
\begin{equation}
\ln(\partial_\tau + \omega_q).
\end{equation}
Since we are taking the trace over periodic states, \(\varphi_\lambda(\tau =0) = \varphi_\lambda(\tau =\beta)\), we introduce the plane wave basis
\begin{align*}
u_{\nu q} &= \frac{1}{\sqrt{\beta}}\e^{i(\vb q\cdot \vb{r} - \omega_\nu\tau)}\\
\omega_\nu &= \frac{2\nu\pi}{\beta}
\end{align*}
The \(\omega_\nu\)'s are the Matsubara boson frequencies. These basis functions are periodic on the interval \( \tau \in [0,\beta) \).
We have
\begin{align*}
\tr\ln(\partial_\tau + \omega) &= \sum_\nu \ev{\ln(\partial_\tau + \omega_q)}{\nu}\\
&=\frac{1}{\beta}\sum_{\omega_\nu}\ln(-i\omega_\nu + \omega_q),
\end{align*}
which in turn implies that
\begin{equation}
\label{eq:boson_partition}
\Z = \e^{-\sum_{q}\sum_{\omega_\nu}\ln(-i\omega_\nu+\omega_q)}
\end{equation}

To compute \eqref{eq:boson_partition}, we need a result for
\begin{equation}
\sum_{\omega_\nu}\ln(-i\omega_\nu +\omega_q).
\end{equation}
Using the same technique as we did in section \ref{sec:free_electron}, we observe that \(i\omega_\nu\) are poles in the Bose-Einstein distribution 
\begin{equation}
b(z) = \frac{1}{\e^{\beta z}-1},
\end{equation}
with \(\Res b(i\omega_\nu) = \flatfrac{1}{\beta} \).
\begin{figure}
	\centering
	\begin{tikzpicture}[scale = 3]
\draw [-, thick] (0, 1) to (2, 1) ;

\draw [-, thick] (1, 0.1) to (1,1.9) ;

\foreach \y in {0.2, 0.4, 0.6, 0.8, 1, 1.2, 1.4, 1.6, 1.8}
	\node at (1, \y) {$\cross$};

\draw[red,thick,dashed,   ->] (1.2, 0.6) to [in = 0,out = 90] (1, 2) to [in = 90, out = 180] (0.8, 1.4);

\draw[red,thick, dashed, ->] (0.8, 1.4) to [in = 180,out = 270] (1, 0) to [in = 270, out = 0] (1.2, 0.6);

\node at (1.4, 1) (e){$\cross$};
\draw[] (e) node[anchor = north west] {$\omega_q$};

\node[anchor = west] at (1.2, 0.4) {$\sim C$};

\node at (2.1,1) {$=$};

%%%%%%%%%%%%%%%%%%%

\draw [-, thick] (2.2, 1) to (4.2, 1) ;

\draw [-, thick] (3.2, 0.1) to (3.2,1.9) ;

\foreach \y in {0.2, 0.4, 0.6, 0.8, 1, 1.2, 1.4, 1.6, 1.8}
	\node at (3.2, \y) {$\cross$};
	
\draw[decoration={markings, mark=at position 0.125 with {\arrow{>}}},
        postaction={decorate}, red, dashed, thick] (3.2, 1) circle (0.08cm);
        
\draw[decoration={markings, mark=at position 0.125 with {\arrow{>}}},
        postaction={decorate}, red, dashed, thick] (2.8, 1.9) parabola bend (3.2, 1.13) (3.6, 1.9);

\draw[decoration={markings, mark=at position 0.125 with {\arrow{>}}},
        postaction={decorate}, red, dashed, thick] (3.6, 0.1) parabola bend (3.2, 0.87) (2.8, 0.1);

\end{tikzpicture}
\end{figure}
\begin{figure}
	\centering
	\begin{tikzpicture}[scale = 3]
\draw [-, thick] (0, 1) to (2, 1) ;

\draw [-, thick] (1, 0.1) to (1,1.9) ;

\node at (0.2, 1.3) {$=$};

\foreach \y in {0.2, 0.4, 0.6, 0.8, 1, 1.2, 1.4, 1.6, 1.8}
	\node at (1, \y) {$\cross$};

\draw[decoration={markings, mark=at position 0.125 with {\arrow{>}}},
        postaction={decorate}, red, dashed, thick] (1, 1) circle (0.08cm);

\node at (1.4, 1) (e){$\cross$};
\draw[] (e) node[anchor = north west] {$\omega_q$};


\node at (2.1,1) {$+$};

%%%%%%%%%%%%%%%%%%%

\draw [-, thick] (2.2, 1) to (4.2, 1) ;

\draw [-, thick] (3.2, 0.1) to (3.2,1.9) ;

\foreach \y in {0.2, 0.4, 0.6, 0.8, 1, 1.2, 1.4, 1.6, 1.8}
	\node at (3.2, \y) {$\cross$};

%%%% over
\draw[decoration={markings, mark=at position 0.425 with {\arrow{>}}},
        postaction={decorate}, red, dashed, thick] (2.2, 1.05) -- (3.1, 1.05);
        
\draw[decoration={markings, mark=at position 0.425 with {\arrow{>}}},
        postaction={decorate}, red, dashed, thick] (3.1, 1.05) to [in = 90, out=90] (3.3, 1.05);


\draw[decoration={markings, mark=at position 0.425 with {\arrow{>}}},
        postaction={decorate}, red, dashed, thick] (3.3, 1.05) -- (4.2, 1.05);        
 
 
%%%% Under 


\draw[decoration={markings, mark=at position 0.425 with {\arrow{>}}},
        postaction={decorate}, red, dashed, thick] (4.2, 0.95) -- (3.3, 0.95);
        
\draw[decoration={markings, mark=at position 0.425 with {\arrow{>}}},
        postaction={decorate}, red, dashed, thick] (3.3, 0.95) to [in = 270, out=270] (3.1, 0.95);


\draw[decoration={markings, mark=at position 0.425 with {\arrow{>}}},
        postaction={decorate}, red, dashed, thick] (3.1, 0.95) -- (2.2, 0.95);        
 


\end{tikzpicture}
\end{figure}
As seen in the figures, the contributions from the pole in the origin cancel. Using this and Cauchy's residue theorem gives, with \(g(i\omega_\nu)=\ln(-i\omega_\nu + \omega_q)\)


\begin{align}
\sum_{\omega_\nu}g(i\omega_\nu) &= +\frac{\beta}{2\pi i}\oint_\mathcal{C}\dd{z} g(z)b(z)\nonumber \\
&= \frac{\beta}{2\pi i}\int_{-\infty}^{0^-}\dd{\varepsilon}b(\varepsilon)\left[\ln(-\varepsilon - i\delta +\omega_q)-\ln(-\varepsilon +i\delta + \omega_q)\right] \nonumber\\
&+ \frac{\beta}{2\pi i}\int_{0^+}^{\omega_q}\dd{\varepsilon}b(\varepsilon)\left[\ln(-\varepsilon - i\delta +\omega_q)-\ln(-\varepsilon +i\delta + \omega_q)\right] \nonumber\\
&= \beta\int_{\omega_q}^\infty \dd{\varepsilon}b(\varepsilon)\left[\ln(-\varepsilon - i\delta +\omega_q)-\ln(-\varepsilon +i\delta + \omega_q)\right]\label{unc:limits_integral}
\end{align}
The contribution from \(\varepsilon <\omega_q\) disappear from the exact same reason as in the case of fermions. The contribution from \(\varepsilon >\omega_q\) is easier, since the difference in the logarithms is \(-2\pi i\), so that 
\begin{align*}
-\beta\int_{\omega_q}^{\infty}\dd{\varepsilon}b(\varepsilon) &= -\beta\int_{\omega_q}^{\infty}\dd{\varepsilon}\frac{\e^{-\beta \varepsilon}}{1-\e^{-\beta \varepsilon}}\\
&= -\frac{\beta}{\beta}\left[\ln\left(1-\e^{-\beta \varepsilon}\right)\right]_{\omega_q}^{\infty} \\
&= \ln\left(1-\e^{-\beta\omega_q}\right).
\end{align*}
We then have
\begin{align}
\Z &= \e^{-\sum_{q}\sum_{\omega_\nu}\ln(-i\omega_\nu+\omega_q)} \nonumber\\
&= \e^{-\sum_q\ln\left(1-\e^{-\beta\omega_q}\right)} \nonumber \\
&= \prod_q \frac{1}{1-\e^{-\beta\omega_q}}\label{eq:Bose_Einstein_Partition}\\
&= \e^{-\beta F}. \nonumber
\end{align}
We recognize \eqref{eq:Bose_Einstein_Partition} as the partition function for a free boson gas, with free energy 
\begin{equation}
F = \frac{1}{\beta}\sum_q\ln\left(1-\e^{-\beta\omega_q}\right).
\end{equation}

The answers we have gotten for both the free fermion gas and free boson gas could easily have been found by simple counting arguments. These calculations have however illustrated what hides behind exact expressions as for example \(\Tr A(\partial_\tau)\). In addition, the methods are familiar for interacting problems, which we will consider later.

\begin{figure}
	\centering
	\input{tex/img/green_feynman}
\end{figure}
