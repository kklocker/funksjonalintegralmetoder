\section{Kort rekapitulering av 2. kvantisering for fermioner og bosoner}

Notasjon: $ \mu = $ sett av kvantetall som bestemmer en én-partikkel tilstand.


\subsection{Mangepartikkelbasis}
\newtheorem{theorem}{Eks}
\begin{theorem}


\begin{align*}
\mu &= (\vec{k}, \sigma):\text{Bølgetall, spinn} \\
\mu &= (i, \sigma) : \text{Gitterpunkt, spinn} \\
\mu &= (n, i) : \text{Orbital, gitterpunkt} \\
\end{align*}

\end{theorem}

En mange-partikkel basis kan skrives $\ket{\phi} = \ket{n_{\mu}, n_{\nu}, \dots, n_{\mu_N}}$. Mangepartikkeltilstander er bygget opp av én-partikkel tilstander, men hvor én-partikkeltilstandene ikke nødvendighvis er uavhengige. Dersom \underline{ett} av kvantetall-settene, $\mu_i$, endres, får denne \underline{spredningen} generelt konsekvenser for fordelingen av kvantetall i de øvrige kvantetallene $\{\mu_j\}_{j\ne i}$.
Vi tenker oss at en generell mange-partikkeltilstand kan bygges opp som en lineærkombinasjon av $\ket{\phi}$-er;
\begin{equation}
\ket{\Psi} = \sum_{n_{\mu},\dots, n_{\mu_N}}\phi_{{\mu},\dots, n_{\mu_N}}\ket{{\mu},\dots, n_{\mu_N}}.
\end{equation}
Én-bestemt tilstandsvektor $\ket{n_{\mu},\dots, n_{\mu_N}}$ kan kreves fra en \underline{vakuum}-tilstand $\ket{0}$ (ingen kvant i noen av de mulige én-partikkeltilstandene) via kreasjons-operatorer.
\begin{align*}
&\textbf{bosoner}: &a_\mu^\dagger \\ 
&\textbf{fermioner}: &c_\mu^\dagger
\end{align*}


Et kvant i en én-partikkeltilstand kan destrueres  v.h.a annihilasjonsoperatorene

\begin{align*}
&\textbf{bosoner}: &a_\mu \\ 
&\textbf{fermioner}: &c_\mu
\end{align*}


Disse operatorene oppfyller diverse kommutasjonsrelasjoner
\begin{align}
&[a_\mu, a_{\nu}] & = [a_\mu^\dagger, a_{\nu}^\dagger] = 0 \\
&[a_\mu, a_{\nu}^\dagger] & = \delta_{\mu\,\nu} \\
&[A,B] &= AB-BA \\
&\{c_\mu^\dagger, c_{\nu}^\dagger\} & = \{c_\mu, c_{\nu\}} = 0 \\
&\{c_\mu, c_{\nu}^\dagger\} & = \delta_{\mu\,\nu} 
\end{align}
I tillegg kommer Pauli-prinsippet, som gir symmetriske/antisymmetriske kommutatorer ved ombytte, avhengig om det er fermion eller boson. 

\subsection{Fra klassisk til 2.kvantisering av én-partikkel operatorer}

For én-partikkel operatorer har vi som regel for den kinetiske energien
\begin{equation}
T = \sum_i T\left(\vec{r}_i, \vec{p}_i\right) = \sum_i T\left(\vec{r}_i, \diffp{}{r}\right)
\end{equation}
\begin{theorem}
Eksternt elektrostatisk potensial:
\begin{equation}
T = \sum_i V_{\text{ext}}\left(\vec{r}_i\right)
\end{equation}
\end{theorem}
\begin{theorem}

Kinetisk energi:

\begin{equation}
T = \sum_i \frac{p^2}{2m} = -\sum_i \frac{\hbar^2}{2m}\nabla_i^2
\end{equation}
\end{theorem}

\begin{theorem}
Krystall-potensial:
\begin{equation}
T = \sum_i \sum_j v_{\text{cryst}} \left( \vec{r}_i, \vec{R}_j \right)
\end{equation}
\end{theorem}

2. kvantisering av en slik operator kan skrives
\begin{equation}
T = \sum_{\mu, \nu} T_{\mu\nu}c_\mu^\dagger c_{\nu},
\end{equation}
hvor 
\begin{equation}
T_{\mu\,\nu} = \mel{\mu}{T\left(\vec{r}, \vec{p}\right)}{\nu}.	
\end{equation}

\textbf{NB: Matriseelementet av én-partikkel operatorer er bestemt av matrise-elementer i Hilbertrommet av én-partikkel tilstander}

\subsection{Fra klassisk til 2. kvantisering av to-partikkel operator}

Vi ser typisk på par-potensialer 
\begin{equation}
V = \sum_{i, j} V\left(\vec{r}_i, \vec{r}_j\right).
\end{equation}

\begin{theorem}
Vekselvirkning mellom to ladninger
\begin{equation}
V = \frac{e^2}{2}\sum_{i\ne j} \frac{1}{|\vec{r}_i - \vec{r}_j|}
\end{equation}
\end{theorem}

2. kvantisert versjon er

\begin{equation}
V = \sum_{\mu, \dots, \beta} V_{\mu\nu\alpha\beta}c_{\mu}^\dagger c_{\nu}^\dagger c_{\alpha}c_{\beta},
\end{equation}
hvor igjen
\begin{equation}
V_{\mu\nu\alpha\beta} = \mel{\mu\nu}{V\left(\vec{r}_i, \vec{r}_j\right)}{\beta\alpha}
\end{equation}

\textbf{NB: Matrise-elementer av to-partikkel operatorer er bestemt av matrise-elementer i Hilbert-rommet av to-partikkel tilstander.}

Hamilton-operatoren:
\begin{align}
H &= T + V \label{eq:hamiltonian} \\
T &= -\sum_i \frac{\hbar^2}{2m}\nabla_i^2
\end{align}

Så langt er dette bare gjort for fermione-operatorer, men helt tilsvarende utsagn gjelder selvsagt for bosoner. 2. kvantisert versjon av Hamilton-operatoren for et vekselvirkende, materielt, boson-system har identisk samme form som \ref{eq:hamiltonian}. Legg merke til at hvert ledd i $H$ har like mange $c_\mu^\dagger$ som $c_{\nu}$.


\begin{figure}
\centering
\begin{tikzpicture}

	\draw[thick, ->- = 0.5] (0,0) -- (1,1);
	\draw[thick, ->- = 0.5] (1, 1) -- (0,2);
	\draw[thick, dashed] (1, 1) -- (2.5, 1);
	\node[thick] at (2.5,1){$\cross$};
	\node[anchor = north west, thick] at (0.5, 0.5) {$\lambda_2$};
	\node[anchor = south west, thick] at (0.5, 1.5) {$\lambda_1$};
	\node[anchor = north west, thick] at (2, 1) {$v_{\lambda_1\lambda_2}$};
	\node[anchor = east] at (0, 1) {$T$:};
\end{tikzpicture}
\caption{Spredning fra eksternt potensial $v_{\mu\nu}c_{\mu}^\dagger c_{\nu}$}
\end{figure}


\begin{figure}
\centering
\begin{tikzpicture}

	\draw[thick, ->- = 0.5] (0,0) -- (1,1);
	\draw[thick, ->- = 0.5] (1, 1) -- (0,2);
	\draw[thick, dashed] (1, 1) -- (2.5, 1);
	\draw[thick, ->- = 0.5] (3.5, 0) -- (2.5, 1);
	\draw[thick, ->- = 0.5] (2.5, 1) -- (3.5, 2);
	
	\node[anchor = north] at (0, 0){$\lambda_3$};
	
	\node[anchor = north] at (0, 2){$\lambda_1$};	
	
	\node[anchor = north] at (3.5, 0){$\lambda_4$};

	\node[anchor = north] at (3.5, 2){$\lambda_2$};
	
	\node[anchor = east] at (0, 1) {$V$:};
\end{tikzpicture}
\caption{Vekselvirkning mellom to partikler.}
\end{figure}