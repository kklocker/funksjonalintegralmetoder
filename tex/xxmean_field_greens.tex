\section{Mean field Green's function}
Physical interpretation if the saddle point. 

\begin{equation}
\psi = 
\begin{pmatrix}
\varphi_{\uparrow} \\
\varphi_\downarrow^\dagger
\end{pmatrix}\qquad 
\psi^\dagger = 
\begin{pmatrix}
\varphi_{\uparrow}^\dagger &
\varphi_\downarrow
\end{pmatrix}
\end{equation}

Green's function 
\begin{align*}
\mathcal{G}_F &= -\ev{\psi\psi^\dagger} \\
&= -\ev{\begin{pmatrix}
	\varphi_{\uparrow} \\
	\varphi_\downarrow^\dagger
	\end{pmatrix}
	\begin{pmatrix}
	\varphi_{\uparrow}^\dagger &
	\varphi_\downarrow
	\end{pmatrix}}\\
&= 
\begin{pmatrix}
-\ev{\varphi_\uparrow\varphi_\uparrow^\dagger} & -\ev{\varphi_\uparrow\varphi_\downarrow} \\
-\ev{\varphi_\downarrow^\dagger\varphi_\uparrow^\dagger} & -\ev{\varphi_\downarrow^\dagger\varphi_\downarrow}
\end{pmatrix} \\
\mathcal{G}_F(k) &= 
\begin{pmatrix}
G_{11}(k) & F(k) \\
F^\dagger(k)& G_{22}(k)
\end{pmatrix} \\
&= \frac{1}{(i\omega_n)^2-\varepsilon_k^2}\cdot\begin{pmatrix}
i\omega_n + \varepsilon_k & a \\
a^\dagger & i\omega_n - \varepsilon_k
\end{pmatrix}
\end{align*}
The Green's function of the fermionic system in the presence of a static boson field that creates and annihilates electron pairs. 
In absence ($a = a^\dagger = 0$):
\begin{align}
\mathcal{G}_F(\vb{k}, i\omega_n) &= 
	\begin{pmatrix}
	\frac{1}{i\omega_n - \varepsilon_k}  & 0 \\
	0 & \frac{1}{i\omega_n + \varepsilon_k} 
	\end{pmatrix}
	\\
	&=
	\begin{pmatrix}
	G_{11} & 0\\
	0 & G_{22}
	\end{pmatrix}
\end{align}
With the particle propagator $G_{11}$ and hole propagator $G_{22}$ which is as for a free electron gas. See Figure \ref{fig:propagators}.
\begin{figure}
	\centering
	\begin{tikzpicture}[scale = 0.75]
	\node[anchor = east] (a) at (0,0) {$F^\dagger = $};
	\node[anchor = east]  at (0,1) {$F = $};
	\node[anchor = east]  at (0,2) {$G_{22} = $};
	\node[anchor = east]  at (0,3) {$G_{11} = $};
	
	\draw[->- = 0.67, thick] (0,3) -- (0.5,3);
	\draw[->- = 0.5, thick] (0.5,3) -- (1,3);
	
	\draw[->- = 0.5, thick] (0.5,2) -- (0,2);
	\draw[->- = 0.67, thick] (1,2) -- (0.5,2);
	
	\draw[->- = 0.5, thick] (0.5,1) -- (0,1);
	\draw[->- = 0.5, thick] (0.5,1)--(1,1);
	
	\draw[->- = 0.67, thick] (0,0)--(0.5,0) ;
	\draw[->- = 0.67, thick] (1,0)--(0.5,0);
	
	
	\draw[decorate,decoration={brace,amplitude=11pt, mirror},xshift=2pt,yshift=0pt]
	(1,-0.1) -- (1,1.1);
	
	
	\node[anchor = west] at (1.4, 0.5) {Anomalous Green's functions. };
	\node[anchor = west] at (1.4, 2) {Hole propagator (exists when $a = 0$)};
	\node[anchor = west] at (1.4, 3) {Particle propagator (exists also when $a^\dagger = 0$)};	
\end{tikzpicture}
	\caption{Propagators of the system}
	\label{fig:propagators}
\end{figure}
$F\sim a, F^\dagger \sim a^\dagger $.
These two functions do not exist in the normal state, since $a = a^\dagger = 0$ in this state. Now, we are able to interpret what it means to have $a \ne 0, a^\dagger \ne 0$.
Notice that 

\begin{align}
\label{eq:order_param}
\begin{split}
\ev{\varphi_\downarrow\varphi_\uparrow} &\sim a\\
\ev{\varphi_\uparrow^\dagger\varphi_\downarrow^\dagger} &\sim a^\dagger
\end{split}
\end{align}
NB! Remember: when we Hubbard-Stratonovich decoupled $S_I$, we used the terms $a^\dagger \varphi_\downarrow\varphi_\uparrow$ and $a\varphi_\uparrow^\dagger\varphi_\downarrow^\dagger$. $a,\, a^\dagger$ are pair field that are conjugated to the order parameters $\ev{\varphi_\downarrow\varphi_\uparrow}$ and $\ev{\varphi_\uparrow^\dagger\varphi_\downarrow^\dagger}$, analog to the case of a spin system in an external magnetic field. This field is a magnetic field that are conjugated to the order parameter of the spin system, which is the magnetization, $\vb{M} \sim \vb{H}$.
The order parameters of a superconductor is as in \eqref{eq:order_param}. When these are nonzero, there is a spontaneously broken symmetry of the problem. 
Returning to $\Ha$;
\begin{equation}
	\Ha = \sum_x\varphi^*\varepsilon(\grad)\varphi + V\sum_x\varphi^*\varphi\varphi^*\varphi
\end{equation}
This model has a continous symmetry
\begin{align*}
	\varphi(x) &\rightarrow \varphi(x)\e^{i\theta(x)} \\
	\Ha &\rightarrow \Ha,
\end{align*}
 which is a $U(1)$-symmetry. However, in $\ev{\varphi_\downarrow\varphi_\uparrow}$, the phases do not cancel, but instead goes to $\ev{\varphi_\downarrow\varphi_\uparrow\e^{2i\theta}}$. If this phase is completely undetermined, this average will be zero. Thus, having $\ev{\varphi_\downarrow\varphi_\uparrow}\ne 0$ must mean that $\theta$ is a known quantity, i.e. the symmetry is spontaneously broken. 
 More generally:
 
 \textit{When we assume a saddle point in a functional integral, and also assume $\Sa_\text{eff}$ is a minimum for finite values of $\ev{a}, \ev{a^\dagger}$, this is quivalent to the assumption of some spontaneous breaking of symmetry (most often).}
 
 Thus: To choose a suitable decoupling scheme, we have to chose the ``right type'' of bosons in the H-S transformations. This choice is decided by the physcis we expect. 
